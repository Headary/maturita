%kat1
\bookname[kralevic dánský]{Hamlet}{William Shakespeare}

\newpart

\parag{Děj díla}{
Duch Hamletova zemřelého otce Hamleta staršího se zjevuje po nocích, čehož chce
Hamlet mladší využít a~popovídat si s~ním. Dozvídá se od něj, že byl zabit Hamletovým strýcem
Claudiem, který si následně bere Hamletovu matku a~stává se králem. Z~Hamletovy touhy
pomstít se začne předstírat šílenství.

Na hrad přijedou potulní herci, čehož Hamlet ihned využívá. Pro potvrzení Claudiova činu
vkládá Hamlet do hry část popisující otcovu smrt. Během představení Claudius odchází, čímž
utvrzuje Hamletovu domněnku.

Toto zjištění se rozhoduje oznámit své matce. Během rozhovoru Hamlet nalezne někoho se
skrývat za závěsem a~v~domnění, že to je Claudius, zabíjí otce jeho lásky Ofélie Polonia.
Ve strachu z~Hamleta ho Claudius posílá do Anglie s~dopisem obsahující pokyn Hamleta
zabít. To ovšem Hamlet během cesty zjišťuje a~přepisuje dopis do formy, aby byli popraveni ti,
kteří dopis doručí.

Hamlet se vrací do Anglie na Oféliin pohřeb. Ta zešílela kvůli otcově smrti a~utonula. Z~touhy
se pomstít se Oféliin bratr Laertes se domlouvá s~Claudiem a~připravují na Hamleta léčku. Ta
spočívá ve vzájemně nevyrovnaném souboji, kdy Laertes dostane ostrý meč napuštěný jedem,
kdežto Hamlet dostane pouze meč tupý. Aby si byl Claudius jistý Hamletovou smrtí, přidá ještě
do Hamletova nápoje jed.

V~boji vyhrává Hamlet. Získává Laertesův meč a~nevědomě ho otravuje. Jakmile se dozví,
že je meč otrávený, zabijí z~pomsty Claudia. Hamletova matka je svědkem celého incidentu
a~vypíjí otrávený nápoj. Mezitím Hamlet umírá na následky zranění a~otravy jedem z~meče.
Jediný naživu zůstává Horacio, dobrý Hamletův přítel, od kterého se celý příběh dozvídáme.
}
\parag{Téma a~motiv}{touha po moci, pomsta, individualizmus, zrada, láska, smrt}
\parag{Časoprostor}{hrad Elsinor, Dánsko, středověk (asi 16./17.~století)}
\parag{Kompoziční výstavba}{prolog + 5 dějství (expozice, kolize, krize, peripetie, katastrofa), chronologický děj}
\parag{Literární druh a~žánr}{drama/tragédie, epika, próza s~občasnou poezií}

\newpart

\parag{Vypravěč}{drama -- není vypravěč}
\paragtable{Postavy}{
\textbf{Hamlet}&hlavní postava, dánský princ, touží po odplatě, složitá psychika,
ctižádostivý, nedůvěřivý sám sobě\\
\textbf{Claudius} & hlavní postava, dánský král, Hamletův strýc a~nevlastní otec, zákeřný,
podlý, zrádce\\
\textbf{Gertruda}&vedlejší postava, Hamletova matka, bere si Claudia po vraždě Hamleta 
staršího, emočně rozdělená na dobro a~zlo, důvěřivá\\
\textbf{Horacio}&vedlejší postava, Hamletův věrný přítel, čestný\\
\textbf{Polonius}&vedlejší postava, nejvyšší komoří, oddaný králi, nedůvěřivý\\
\textbf{Ofélie}&vedlejší postava, Hamletova láska, Poloniova dcera, zešílí a~utone,
oddaná otci, citlivá\\
\textbf{Laert}&vedlejší postava, bratr Ofélie a~syn Polonia, snaží se pomstít otce
a~sestru\\
}

\parag{Vyprávěcí způsob}
{dvojsmysly, satira (Hodno uvážení; padouch mi otce zabije, já za to co
jediný syn toho padoucha odešlu do nebe?), patos, polopřímá řeč, archaické 
básnické výrazy (nepropůjč, šat)}

\parag{Typy promluv}{dialogy; dlouhé monology; vážná, šílená}
\parag{Veršová výstavba}{blankvers}

\newpart

\parag{Jazykové prostředky a~jejich funkce}
{metafory (modro se vine), sarkazmus (Kde je Polonius? U~večeře\dots), 
nerýmovaný verš, personifikace (Ztiš se, srdce!), obrazná pojmenování 
(Volá tě čas.), spisovný i~nespisovný jazyk, epiteton (růžový večer), 
oxymóron (s~plesáním teskným), inverze, komické i~tragické prvky}

\parag{Tropy a~figury a~jejich funkce}{metafory, ironie, eufemismus}
\parag{Kontext autorovy tvorby}{přelom 16. a~17. století, renesance, 
2. etapa autorovy tvorby}

\parag{William Shakespeare}{
\begin{itemize}
\setlength\itemsep{0em}
\item 1564 -- 1616
\item významný anglický básník, dramatik, herec
\item 37 her
\item Romeo a~Julie, Král Lear 
\end{itemize}
}

\paragtable{Literární/obecně kulturní zasazení}{
\textbf{Thomas Mann}&Doktor Faust\\
\textbf{Francesco Petrarca}&Zpěvník\\
\textbf{Dante Alighieri}&Božská komedie\\
\textbf{Giovanni Boccaccio}&Dekameron\\
}







