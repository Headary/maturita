%kat4
\bookname{Dlouhý, Široký a~Krátkozraký}
\booksubname{}
\bookauthor{Ladislav Smoljak a~Zdeněk Svěrák}

\bookcontent{

\newpart

\parag{Děj díla}{
Princ Jasoň se dozví o~princezně Zlatovlásce, která je zakleta obrem Kolodějem,
a~rozhodne se ji vysvobodit. Jeho bratr Drsoň se taktéž ucházel o~její ruku, ovšem pouze pro bohatství.

Dostane se na hrad, pozná princeznu a~krále a~společně se vydají za obrem Kolodějem. Na své cestě ovšem 
potkají Bystrozrakého, který kvůli učení cizích jazyků ztratil svůj bystrý zrak.

Bystrozraký se k~nim připojuje a~pomáhá jim na cestě.
Na radu Bystrozrakého navštíví ze všeho nejdříve Vševěda, aby jim poradil, jak na obra.
Dozví se, že musí sebral obrovi prsten, jímž se posléze musí $3\times$ dotknout princezny, aby ji osvobodili.

Najdou obra a~princ Jasoň se snaží sundat Kolodějovi prsten z~ruky pomocí kbelíku s~vodou s~mýdlem. 
To je ovšem neúspěšné a~obr zabalí prsten do kapesníku a~strčí si ho do kapsy. Naši hrdinové ovšem 
vymyslím léčku, kterou obra rozbrečí, ten se vysmrká a~při vytahování kapesníku mu vypadne prsten.
Tak obr přichází o~svoji sílu a~Jasoň přeměňuje princeznu zpět na krásnou ženu.
}
\parag{Téma a~motiv}{láska, čestnost, touha po penězích, chamtivost, zlost, parodie původní pohádky}
\parag{Časoprostor}{děj probíhá na jevišti, nezasazeno do reality, lokace - les, královský zámek}
\parag{Kompoziční výstavba}{chronologický děj, 9 obrazů, parodie}
\parag{Literární druh a~žánr}{drama, komedie/pohádka, próza}

\newpart

\parag{Vypravěč}{drama~-- nemá vypravěče}
\parag{Postavy}
\begin{compactdesc}
\item[Zlatovláska] princezna, zakletá Kolodějem $\rightarrow$ ošklivá, původně nádherná
\item[král] otec Zlatovlásky, zoufalý, smutný
\item[princ Jasoň] dobrý, dobromyslný, odvážný, chce vysvobodit princeznu, milý k~lidem
\item[princ Drsoň] zlý, chamtivý, hamižný, chce princeznu pro bohatství
\item[Bystrozraký] původně člen známe trojice, stal se z~něj krátkozraký, pomáhá princi a~hlavním postavám
\item[obr Koloděj] obr, smutný, zklamaný, chce se oženit
\end{compactdesc}


\parag{Vyprávěcí způsob}{
přímá řeč (Žádné chmury, žádné mraky, vede vás Dlouhý, Široký a~Bystrozraký!)
}

\parag{Typy promluv}{dialog, scénické poznámky}

\newpart

\parag{Jazykové prostředky a~jejich funkce}{
oslovování publika (Kampak se nám ti čerchmanti
poděli? Neschovali se, děti. Umřeli!) hovorové výrazy(hukáž, hupejpání), přirovnání (A~teď je
bledá jako smrt.), nářečí(Jsem Bystrozrakyj, hdehdo huž mě z~dálky pozná.), vulgarismy(Sakra),
dvojsmysly (Ten člověk je k~nezaplacení.)
}

\parag{Tropy a~figury a~jejich funkce}{řečnické otázky, metafory, ironie}
\parag{Kontext autorovy tvorby}{doba vlády komunistické strany v~bývalém Československu, rok~1974}

\parag{Ladislav Smoljak}{
\begin{compactitem}
\item 9.~prosince~1931 -- 6.~června~2010
\item scénárista, režisér, herec
\item vystudoval matematiku a~fyziku na Vysoké škole pedagogické
\item zakladatel Divadla Járy Cimrmana
\item Jáchyme, hoď ho do stroje!, Na samotě u~lesa, Vyšetřování ztráty třídní knihy
\end{compactitem}
}


\parag{Zdeněk Svěrák}{
\begin{compactitem}
\item narozen 28.~března~1936
\item dramatik, scénárista, herec, skladatel, spisovatel
\item zakladatel Divadla Járy Cimrmana
\item spolupráce se Smoljakem nebo Uhlířem
\item Obecná škola, Elektrický valčík (píseň), Tři veteráni, Vesničko má středisková
\end{compactitem}
}

\parag{Literární/obecně kulturní zasazení}{
\begin{compactdesc}
\item[Jan Werich] Fimfárum
\item[Karel Poláček] Bylo nás pět
\item[Viktor Dyk] Krysař
\item[Jaroslav Hašek] Osudy dobrého vojáka Švejka za světové války
\end{compactdesc}
}
}
