%kat3
\bookname{Petr a Lucie}
\booksubname{}
\bookauthor{Romain Rolland}

\bookcontent{
\newpart

\parag{Děj díla}
Petr, syn soudce, při náletu náhodně setká v metru s mladičkou Lucií -- stojí
vedle sebe a bezděčně se chytí za ruce Během tohoto krátkého setkání vznikne
mezi oběma náklonnost. Jejich setkání je velmi krátké, ale oběma utkví
v~paměti.

Až později se znovu setkávají na nábřeží, mají velkou radost. Začnou se
pravidelně potkávat a postupně náklonnost přerůstá v opravdový cit a zamilují
se do sebe. Jejich vztah je ještě umocněn hrozbou neustálého nebezpečí.

Ale tíživé problémy války ustupují do pozadí a~oba prožívají chvíle štěstí.
Plánují společnou budoucnost. Jejich láska plná snů je však jen dočasná -- Petr
bude totiž za půl roku povolán do armády. Proto se rozhodnou, že se zasnoubí.
Ale den, kdy k tomu má dojít je pro ně osudný. Na Velký pátek jdou do chrámu
sv.~Gervasia, a za poslechu gregoriánských chorálů oba umírají v troskách
chrámu, který se na ně zřítil (den bombardování Paříže).

\parag{Téma a~motiv}
láska a prostý život v době války; láska, vztah, chudoba vs bohatství, rodinné vztahy, porozumění, strach
\parag{Časoprostor}
1. světová válka (30.~ledna -- 29.~března 1918), Francie -- Paříž
\parag{Kompoziční výstavba} % chronologická, retrospektivní...
chronologická, bez kapitol, členěno na menší celky
\parag{Literární druh a~žánr} % lyrika/epika, poezie/próza...
epika, próza, válečná novela

\newpart

\parag{Vypravěč}
er forma, vševědoucí vypravěč % er forma, neosobní vypravěč
\parag{Postavy}
\begin{compactdesc}
	\item[Petr Aubier] syn vyšší vrstvy, proti válce, vkládá naděje do vztahu
		s~Lucií, ignorování blížícího se nástupu do války
	\item[Lucie] něžná, prostá, citlivá; nedostudovaná malířka, žije pouze
		s~matkou, smyslem života láska
	\item[Filip] Petrův bratr, sebestředný z války, po návratu si nerozumí,
		chápe Petrův vztah
	\item[Luciina matka] rozvedená, čeká se svým novým přítelem dítě, pracuje
		v~továrně na střelivo
\end{compactdesc}

\parag{Vyprávěcí způsob} % řeč přímá, polopřímá, nevlastní přímá, nepřímá
přímá řeč

\parag{Typy promluv} % monolog, dialog, pásmo vypravěče
dialogy, pásmo vypravěče
%\parag{Veršová výstavba}

\newpart

\parag{Jazykové prostředky a~jejich funkce} % spisovný/nespisovný jazyk, hovorový
spisovný jazyk, bohatá slovní zásoba, jednoduché věty
\parag{Tropy a~figury a~jejich funkce} % metafora, metonymie, přirovnání, personifikace
řečnické otázky, metafora (\uv{Vlna smrti ho již zanedlouho odnese\dots}),
přirovnání, personifikace, epitetony (\uv{chvějící se hvězdy}), cizí slova

\parag{Kontext autorovy tvorby}
\begin{compactitem}
	\item vydání v roce 1920 -- po 1.~světové válce
	\item literatura 1.~pol. 20.~stol.~-- protiválečná, ztracená generace
	\item kritika války, popis hrůz, utrpení lidí během války\dots
	\item ztráta amerického snu (tvrdou prací a pílí dosáhneme vrcholu)
\end{compactitem}

\parag{\getauthor}
\begin{compactitem}
	\item 1866--1944
	\item francouzský prozaik, dramatik, esejista
	\item zabýval se i dějinami divadla a umění, přednášel dějiny umění
	\item nositel Nobelovy ceny za literaturu (dílo Jan Kryštof)
	\item antimilitarista, antifašista
	\item Jan Kryštof, Dobrý člověk ještě žije, Okouzlená duše
\end{compactitem}

\parag{Literární/obecně kulturní zasazení}
\begin{compactdesc}
	\item[Ivan Olbracht] Nikola Šuhaj loupežník
	\item[Jaroslav Hašek] Osudy dobrého vojáka Švejka za světové války
	\item[Erich Maria Remarque] Na západní frontě klid
	\item[Ernest Hemingway] Sbohem armádo!
	\item[Francis Scott Fitzgerald] Na prahu ráje
\end{compactdesc}
}
