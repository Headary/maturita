%kat1
\bookname{Zkrocení zlé ženy}
\booksubname{}
\bookauthor{William Shakespeare}

\bookcontent{
\newpart

\parag{Děj díla}
Lucentio přijede studovat do Padovy, kde se právě konají oslavy, na kterých
pozná Bianku. Na první pohled se do ní zamiluje. Bianka má několik nápadníků,
ale její otec chce nejdříve provdat starší dceru Kateřinu. Do Padovy přijede
Hortenziův přítel Petruccio, a~chce zde najít manželku. Hortenzio se domluví
s~Petrucciem, aby si vzal Kateřinu. Druhého dne jde Petruccio na námluvy.
Kateřina zuří jako obvykle, a~tak si Petruccio řekne, že ať mu bude nadávat,
jak chce, ať ho bude vyhánět, bude ji říkat jak je hodná, milá, výřečná. A~tak
se začne připravovat svatba. Petruccio odjede do Benátek. V~neděli se koná
svatba. Všichni obyvatelé Padovy se sejdou u~kostela, ale ženich nikde.
Padované se jí smějí, že takovou saň si nechce nikdo vzít a~tak jí ženich
utekl. Petruccio nakonec přijede, na starém koni v~rozedraném oblečení.
Kateřina zuří. Petruccio se po celou dobu chová jako hulvát a~to je i~na
Kateřinu moc. Svatba i~přes neustálé Kateřininy protesty proběhne. Uprostřed
svatebního veselí chce Petruccio odjet, ač ho všichni prosí, aby zůstal, odjede
i~s~protestující Kateřinou. Po dlouhé namáhavé cestě dojedou konečně do
Petrucciova domu. Petruccio nařídí sluhům, aby přinesli večeři. Jakmile chce
ale Kateřina jíst, je všechno špatně -- jídlo je připálené, nedovařené,
nechutné atd.  Když přijdou do ložnice, je tam špatně ustlaná postel. Druhého
dne nechá Kateřina uklidit a~upravit Petrucciův dům. Mezitím se v~Padově
Lucentiův sluha v~přestrojení za Lucentia domluví Lucentiův sňatek s~Biankou.
Biancin otec k~sňatku svolí a~na svatební veselí pozve i~Kateřinu a~Petruccia.
Petruccio sezve krejčí, aby mu předvedli šaty pro Kateřinu podle nejnovější
módy. A~všechny mají nějakou vadu. Kateřině to rve srdce. Nakonec ale kateřinu
zkrotí a~je poslušná jak beránek. Na svatbě se ale provalí, že se Lucentio
a~jeho sluha \uv{vyměnili}, aby se mohl Lucentio dvořit Biance. A~tak se konala
šťastná svatba. Při svatebním veselí se Petruccio, Lucentio a~Hortensio vsadí,
která z~jejich žen je nejposlušnější. První zavolá pro svou choť Lucentio,
jenže Bianka mu vzkáže, že má práci a~nemůže přijít. Hortenzio zavolá jako
druhý, ale jeho manželka mu odpoví, že se jí nechce, ať přijde za ní. Nakonec
poručí Petruccio, aby Kateřina přišla. Kateřina přijde a~dovede i~obě zbylé
volané manželky. Všichni jsou Kateřininým chováním překvapeni.

\parag{Téma a~motiv}
postavení a~neposlušnost ženy, zlomení drzosti a~převýchova, ukazuje zde normu
společnosti vnímání rovnosti partnerů a~požadavky na budoucí choť, dobývání
\parag{Časoprostor}
italské město Padova, autorova současnost (16.~století)
\parag{Kompoziční výstavba} % chronologická, retrospektivní...
5 dějství; chronologická výstavba
\parag{Literární druh a~žánr} % lyrika/epika, poezie/próza...
drama, poezie, komedie

\newpart

\parag{Vypravěč} % er forma, neosobní vypravěč
absence vypravěče
\parag{Postavy}
\begin{description}
	\item[Kateřina] sestra, nelze provdat, sebevědomá, nepodvoluje se mužům, drzá, hrubá, výbušná
	\item[Bianca] mladší sestra Kateřiny, mírná, laskavá, poddaná, milá, hodná
	\item[Gremio a~Hortensio] nápadníci Bianky
	\item[Baptista Minola] bohatý šlechtic z~Padovy, otec Biancy a~Kateřiny, štědrý
	\item[Petruccio] veroský šlechtic, vezme si Kateřinu a~utýrá ji do podoby poslušné pokorné ženy
	\item[Lucensio] zamiluje se do Biancy, lstí se stane jejím učitelem, veronský šlechtic
	\item[Vincenzio] otec Lucencia
	\item[Tranio a~Biondello] slouží Lucensio
	\item[Grumio a~Curzio] slouží Petrucciovi
\end{description}

\parag{Vyprávěcí způsob} % řeč přímá, polopřímá, nevlastní přímá, nepřímá
přímá řeč
\parag{Typy promluv} % monolog, dialog, pásmo vypravěče
monology, dialogy, scénické poznámky
\parag{Veršová výstavba}
blankvers

\newpart

\parag{Jazykové prostředky a~jejich funkce} % spisovný/nespisovný jazyk, hovorový
spisovný jazyk, verše, expresivní výrazy, poetismy, historismy, archaismy
\parag{Tropy a~figury a~jejich funkce} % metafora, metonymie, přirovnání, personifikace
eufemismy, dysfemismy, řečnické otázky
\parag{Kontext autorovy tvorby}
\begin{compactitem}
	\item renesance~-- návrat k~antice
	\begin{compactitem}
		\item člověk, život na zemi
		\item vzdělání
		\item individualita
	\end{compactitem}
	\item 16.~a~17.~století
	\item vyjadřování života lidí, komedie v~tragédii a~obráceně
\end{compactitem}

\parag{\getauthor}{
\begin{compactitem}
\item 1564--1616
\item významný anglický básník, dramatik, herec
\item 37 her
\item 3 období psaní: komedie, tragédie, pohádky
\item Romeo a~Julie, Král Lear
\end{compactitem}
}

\parag{Literární/obecně kulturní zasazení}
\begin{description}
\item[Thomas Mann] Doktor Faust
\item[Francesco Petrarca] Zpěvník
\item[Dante Alighieri] Božská komedie
\item[Giovanni Boccaccio] Dekameron
\end{description}
}
