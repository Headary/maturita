%kat3
\bookname{Společenstvo Prstenu}
\booksubname{Pán prstenů}
\bookauthor{John Ronald Reuel Tolkien}

\bookcontent{

\newpart

\parag{Děj díla}{
	Kniha začíná oslavou narozenin Bilba Pytlíka, hlavního hrdiny
	knihy \emph{Hobit}, a~jeho synovce (bratrance) Froda, kterého adoptoval,
	když se jeho rodiče utopili, a~který má shodou okolností narozeniny ve
	stejný den. Bilbo chce odejít z Kraje a~zažít ještě pár dobrodružství a pro
	odchod z oslavy si vybere mimořádný způsob, zmizí pomocí kouzelného
	prstenu, který na své minulé cestě získal. Ve své noře setká s Gandalfem,
	čarodějem a~starým přítelem, jenž se také přijel podívat na
	Bilbovy narozeniny. Gandalf má podezření na to, že prsten Bilbovi ubližuje
	a~tak jej nutí ho přenechat Frodovi. Po krátké slovní roztržce Gandalf
	Bilba přesvědčí a~Bilbo odchází. O~chvíli později přijde i~Frodo a~Gandalf
	mu prsten předá s tím, že jej radši nemá používat a~že si musí něco
	prověřit. Gandalf poté zjistí, že onen kouzelný prsten Bilbův je vlastně
	Jeden vládnoucí prsten Temného pána Saurona a~že ten už o~něm ví díky
	Glumovi, tvorovi, kterému Bilbo prsten sebral a~kterého při jeho bloudění
	zajali Sauronovi služebníci, a~poslal své nejstrašnější posluhovače,
	Nazgůly, Jeden prsten najít. Gandalf řekne Frodovi, že se musí připravit
	a~pokud v brzké době nepřijde, tak se v den svých narozenin vypravit do
	Hůrky, kde se opět setkají. Poté odjede za Sarumanem, hlavou řádu čarodějů,
	aby se poradil, ale tam se dozví, že Saruman začal také toužit po prstenu,
	a~je jím zajat ve věži Orthanku. Frodo mezitím vyčkává a~když Gandalf po
	dlouhé době nepřichází, prodá svojí zděděnou noru Dno Pytle a~s pomocí
	svých druhů předstírá, že se stěhuje do Rádovska.

	Večer, když už skupinka odchází, přijíždí do Hobitína černý jezdec
	a~vyptává se na Pytlíka, ale je mu řečeno, že už se odstěhoval. Frodo
	s ostatními pak putuje lesem, setká se s lesními elfy, několikrát těsně
	unikne černým jezdcům, ve Starém Hvozdě si odpočine u~Toma Bombadila, který
	je vysvobodí ze spárů Dědka Vrbáka a~poté je ještě jednou zachrání na
	Mohylových vrších. Po těchto událostech dorazí skupinka konečně do Hůrky,
	kde je však čeká další šok v podobě sdělení hostinského, že Gandalf už se
	víc jak půl roku neukázal. Potkají zde však hraničáře, kterému lidé říkají
	Chodec, díky němuž v noci opět uniknou černým jezdcům a~který je pak vede
	dále. Přicházejí tak na Větrov, kde jsou černými jezdci napadeni a~Frodo je
	zasažen čepelí jejich kapitána. Čeká je ještě 14 dní do Roklinky, kam mají
	nyní namířeno a~Frodo vypadá zle. Naštěstí potkají elfa Glorfindela, jenž
	je měl za úkol hledat a~ten posílá svého koně s Frodem úprkem do Roklinky.
	Těsně před Roklinkou u~Bruinenského brodu je pak znovu dohnán černými
	jezdci, ale Elrond spolu s Gandalfem, který mezitím uprchl z Orthanku na
	křídlech orla Gwaihira, vyvolají obrovskou vlnu, která připomíná jedoucí
	koně a~černí jezdci jsou jí smeteni. Frodo se tak po strastiplné cestě
	dostane do Roklinky, kde se setká s Gandalfem a~kde se koná Elrondova rada,
	na kterou se sjeli zástupci všech svobodných národů Středozemě a~kde musí
	být rozhodnut osud prstenu, který zde nemůže zůstat. Je rozhodnuto, že
	prsten musí být odnést tam, kde byl vykován a~kde jedině může být zničen,
	do Puklin Osudu v zemi Mordor. Nastává otázka, kdo prsten ponese a~jelikož
	se ostatní začnou hádat, vezme na sebe toto břemeno Frodo. Je mu vyvoleno
	osm společníků, kteří dohromady s ním tvoří Společenstvo prstenu. Za lidi
	Chodec~-- Aragorn a~Boromir, syn správce Gondoru, za elfy Legolas, syn
	krále Thranduila z Lesní říše, za trpaslíky Gimli, syn Glóina a~nakonec
	jeho největší přátelé, Gandalf a~tři hobiti, Samvěd, Smíšek a~Pipin. Tato
	skupina se pak vydává na jih, je napadena vlky a~po neúspěšném pokusu
	o~průchod průsmykem Caradhasu se vydává do Morie, dávného sídla trpaslíků,
	kde však nyní podle pověstí sídlí strašná příšera. Před jejími branami jsou
	napadeni hlídačem ve vodě a~o~pár dní později skřety. Utíkají pryč, ale
	když už jsou téměř u~druhé brány, objeví se ona příšera, Balrog, démon
	pradávného světa. Gandalf se s ním utká na můstku a~svrhne ho dolů.
	Naneštěstí však Balrog Gandalfa stáhne s sebou a~tak je otřesená skupina
	nucena pokračovat dále bez Gandalfova vedení, kterého se ujímá Aragorn. Tak
	dorazí do skryté elfí říše Lothlórienu, kde jsou po počátečních problémech
	přivítání a~kde si chvíli odpočinou. Pak se vydávají po řece Anduině směrem
	k Raurorským vodopádům a~Aragorn stále neví, co bude dál. Projedou branou
	králů Argonathem, po přistání na palouku Parth Galenu se však situace
	vyvinula sama. Boromira prsten ovládl a~ten ho chtěl Frodovi vzít. Kvůli
	této události se Frodo rozhodne hned odejít a~vydat se na pouť do Mordoru
	sám. Mezitím Sam najde Froda a~na poslední chvíli se mu podaří se za ním
	dostat a~přeplouvají tak spolu na druhý břeh Anduiny.
}

\parag{Téma a~motiv}{hrdinství, boj dobra a~zla, magično/nadpřirozeno, smyšlený svět}
\parag{Časoprostor}{fiktivní země Středozemě (Kraj, Roklinka, Morie, Lórien,
Fangorn\dots), 3001~-- 3019 Třetího věku}
\parag{Kompoziční výstavba}{chronologická, lineární; 4 části~-- předmluva, prolog, kniha první a~kniha druhá}
\parag{Literární druh a~žánr}{epika, próza; fantasy román}

\newpart

\parag{Vypravěč}{er forma}
\parag{Postavy}{
	\begin{description}
		\item[Frodo Pytlík] hlavní hrdina příběhu, hobit, nositel prstenu, odvážný,
			laskavý, duševně silný (dlouho odolává moci prstenu, avšak ten mu postupně
			zatemňuje mysl~-- zvýšená agresivita a~nenávist)
		\item[Samvěd Křepelka] hobit, věrný, pokorný, statečný (když není zbití),
			Frodův nejlepší přítel
		\item[Peregrin Bral (Pipin)] hobit, odvážný, trochu hloupý, Frodův přítel
		\item[Smělmír Brandorád (Smíšek)] hobit, odvážný, Frodův přítel, s pipinem
			odlehčují vážnost situací příběhu
		\item[Gandalf Šedý (Mithrandir) ] čaroděj, moudrý, laskavý, vůdce společenstva,
			člen rasy Maiar, nemá záporné vlastnosti, slouží jako symbol naděje
		\item[Legolas z lesní říše] vysoký a~vznešený elf, potomek starého elfského
			krále, výborný lučištník a~stopař, Aragornův přítel
		\item[Gimli syn Gloinův] trpaslík, silný, odolný a~statečný, tvrdohlavý,
			výborný v boji se sekyrou
		\item[Aragorn syn Arathornův] statečný a~moudrý hraničář, má mnoho
			pseudonymů~-- Chodec, Dúnadan, Isildurův dědic~-- je právoplatným
			dědicem Gondorského království, po Gandalfově pádu vede
			společenstvo, skvělý v boji s~mečem
		\item[Boromir z Gondoru] silný, bojovný, statečný, hrdý generál Gondorských
			vojsk, nejstarší syn Denetora~-- správce Gondoru, nechá se zlákat mocí
			prstenu a~snaží se ho Frodovi vzít, nakonec obětuje svůj život při záchraně
			Pipina a~Smíška
		\item[Arwen] dcera Elronda, elfka, krásná a~laskavá, miluje Aragorna
		\item[Galadriel] elfka, paní Lórienu, jedna z nejstarších a~nejvznešenějších
			obyvatel Středozemě, nositelka jednoho ze 3 elfských prstenů moci
		\item[Saruman Bílý] nejvyšší čaroděj, spojenec Saurona a~Mordoru, snaží se
			získat prsten, později označovaný a~Sarumana Barevného
		\item[Elrond z Roklinky] půlelf, velmi starý a~moudrý, vládce Roklinky,
			vlastní jeden z elfských prstenů moci
		\item[Glum (Smeagol)]
		\item[Sauron] hlavní záporná postava, symbol všeho zla, ten, co stvořil
			prsteny moci, snaží se získat vládu nad Středozemí
		\item[Nazgúlové] přezdívaní také jako \uv{Devítka}, původně lidé; dostali od
			Saurona prsteny moci a~jeden po druhém se obrátili ke zlu, stali se
			přízraky a~teď slouží Sauronovi; Vůdce Nazgúlů je známý jako černokněžný
			král Angmaru
		\item[Tom Bombadil] nejstarší \uv{tvor} ve Středozemi, jeho moc není zcela
			známa
		\item[Balrog] démon děsu, zosobnění smrti, z rasy Maiar~-- velmi staré
			stvoření, původně sloužící Morghotovi
		\item[Bilbo Pytlík] hobit, Frodův strýc, vlastnil prsten před Frodem
	\end{description}
}

\parag{Vyprávěcí způsob}
{spisovný, vymyšlené výrazy (vrrci, skřeti\dots), archaizmy, knižní výrazy,
jazyky (elfština), vlastní geografie i~historie, písně}

\parag{Typy promluv}{přímá i~nepřímá řeč, dialogy}
%\parag{Veršová výstavba}{blankvers}

\newpart

\parag{Jazykové prostředky a~jejich funkce}
{spisovný jazyk, archaizmy, neologismy}

\parag{Tropy a~figury a~jejich funkce}{personifikace (kopce počaly tísnit; sen
se zmocnil), oxymóron (hledat cestičky v~bezcestí), metafora (vítr začal
vylévat vodu dalekých moří; svahy se vršily), metonymie (temné hlavy kopců),
básnické přívlastky~-- epiteton (mrazivý chlad; daleká moře)}

\parag{Kontext autorovy tvorby}{1. polovina 20. století, doba krátce před druhou světovou válkou,
základ moderní hrdinské fantasy}

\parag{\getauthor}{
\begin{compactitem}
\item 3.~ledna~1892 -- 2.~září~1973
\item britský spisovatel, filolog a~univerzitní profesor
\item považován za otce moderní hrdinské fantasy
\item vytváří umělé země, kultury, jazyky, písma
\item další díla: Pán prstenů, Silmarilion, Příběhy Toma Bombadila
\end{compactitem}
}

\parag{Literární/obecně kulturní zasazení}
\begin{description}
\item[C. S. Lewis] Letopisy Narnie
\item[J. M. Barrie] Petr Pan
\item[R. E. Howard] Povídky o~barbaru Conanovi
\end{description}
}
