%kat2
\bookname{Podivný případ dr.~Jekylla a pana Hyda}
\booksubname{}
\bookauthor{Robert Louis Stevenson}

\bookcontent{

\newpart

\parag{Děj díla}
Právník Mr. Utterson je dlouholetým přítelem doktora Jekylla, proto ho velmi
překvapí, když si přečte Jekyllovu závěť -- stojí v ní, že pokud doktor zmizí a
do 3 měsíců se neobjeví, odkazuje své jmění svému příteli Edwardu Hydovi.
Utterson je ještě zmatenější, když zjistí, kdo onen Hyde je -- při pohledu
na něj tuhne krev v žilách, jde z něj strach. Utterson dokonce zjistí, že Hyde
stojí i za vraždami v Londýně.

Doktor Jekyll je najednou bledý, nejistý, odmítá přijímat návštěvy a po jeho
příteli není ani stopy. Během dalších dvou měsíců se Jekyll chová jako za
starých časů, zdá se být vše v pořádku. Jenže onemocní třetí z přátel -- doktor
Lanyon. Při zmínce o Jekyllovi se mu začínají třást ruce, odmítá se o něm
bavit. Lanyon předá Uttersonovi obálku, protože si je jist, že velmi brzy
zemře. Obálka má být otevřena po jeho smrti a zdá se být klíčem k celé
záhadě.

Lanyon opravdu zemře, Jekyll se zamkne ve své studovně, nepřijímá návštěvy a i
jeho sluha Poole je vystrašený. Tvrdí, že jeho pán musí být po smrti a vrah je
zamčený ve studovně, protože hlas zevnitř není hlas jeho pána! Poole se obrátí
na Uttersona a rozhodnou se pro krajní řešení -- vyrazí dveře studovny a najdou
mrtvého Edwarda Hyda. Marně hledají Jekyllovo tělo nebo jakoukoliv zmínku o
něm.

K obálce od Lanyona přibývá další -- Jekyllova. Právník je pozorně pročítá a
najednou je mu vše jasné. Doktor Jekyll byl vážený a známý vědec, musel se
proto i tak chovat, měl komplexy z nedostatečného dětství, toužil být někým
jiným. Vyvinul látku, po které se jeho osobnost rozdvojuje na Jekylla --
vědce a Edwarda Hyda -- zosobněné zlo.

S experimentem je spokojen, ale naneštěstí se mu začíná vymykat z rukou -- mění
se na Hyda zčistajasna, bez požití drogy. Jeho dvě osobnosti se mezi sebou
nenávidí a častěji nyní vítězí Hyde! Jekyll se rozhodne se vším skoncovat, ale
potřebuje Lanyonovu pomoc. Chudák doktor toto hrůzné objevení neunese. Jekyll
je bezmocný -- suroviny od dodavatele chemika jsou jiné, nevyhovují. Proto
se zamyká ve studovně, aby Hyde už nemohl napáchat další zlo.

\parag{Téma a~motiv}
Vnitřní boj dobra a zla, zachycení lidských stránek; emoce, charakter, zlo,
opuštěnost, preparáty, změna\dots
\parag{Časoprostor}
konec 19. století, Anglie -- Londýn
\parag{Kompoziční výstavba} % chronologická, retrospektivní...
chronologická s retrospektivními prvky; 10 kapitol, 9. retrospektivní
\parag{Literární druh a~žánr} % lyrika/epika, poezie/próza...
epika, próza, sci-fi novela

\newpart

\parag{Vypravěč}{er forma, neosobní vypravěč}
\parag{Postavy}
\begin{compactdesc}
	\item[Gabriel John Utterson] hlavní postava, advokát, píše závěť
		dr.~Jekyllovi; studený, necitlivý, sebekritický, skromný
	\item[dr. Henry Jekyll] lékař a vědec, úžasný člověk, filantrop, chytrý,
		vzdělaný, ctnostný
	\item[Edward Hyde] alterego Jekylla – přímý opak Jekylla, pouze Jekyllovi
		špatné stránky a umocněny
	\item[dr. Hastie Lanyon] lékař, přítel, hádá se s Jekyllem o nevědeckých
		tezích a experimentech 
	\item[Pool] – sluha Jekylla, věrný, důvěryhodný, poctivý
\end{compactdesc}


\parag{Vyprávěcí způsob}
přímá řeč

\parag{Typy promluv}
dialogy, občasný monolog (závěti, zprávy\dots), pásmo vypravěče
%\parag{Veršová výstavba}

\newpart

\parag{Jazykové prostředky a~jejich funkce}
spisovný jazyk, hovorový, archaismy, termíny, jednoduchý jazyk


\parag{Tropy a~figury a~jejich funkce}
přirovnání, metafory, personifikace, epiteton (neskonalou vděčnost)

\parag{Kontext autorovy tvorby}
\begin{compactitem}
	\item 2.~polovina 19.~století
	\item období realismu
		\begin{compactitem}
			\item kritický pohled skutečnosti
			\item úsilí o pravdivé zobrazení skutečnosti
			\item surový, oproštění od příkras a iluzí
		\end{compactitem}
	\item prvky sci-fi
		\begin{compactitem}
			\item vznik v polovině 19. století
			\item moderních technologie, neznámé přírodní jevy
			\item často v vesmíru, budoucnosti nebo alternativní minulosti
		\end{compactitem}
\end{compactitem}

\parag{\getauthor}
\begin{compactitem}
	\item 1850--1894
	\item skotský básník, spisovatel (cestopisy)
	\item představitel novoromantismus v anglické literatuře
		\begin{compactitem}
			\item logicky navazující, racionalita
			\item psáno bez emocí, emoce vyvolány až u čtenáře
			\item odstup vyprávěče
		\end{compactitem}
	\item působení v Edinburgu
	\item přestěhování do Švýcarska kvůli slabým plicím
	\item Ostrov pokladů, Únos
\end{compactitem}

\parag{Literární/obecně kulturní zasazení}
\begin{compactdesc}
		\item[Charles Dickens] Oliver Twist
		\item[Charlotte Brontëová] Jana Eyrová
		\item[Nikolaj Vasilijevič Godol] Revizor
\end{compactdesc}
}
