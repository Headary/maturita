%kat3
\bookname{Malý princ}
\booksubname{Le petit prince}
\bookauthor{Antoine de Saint-Exupéry}

\bookcontent{

\newpart

\parag{Děj díla}{
Pilot vypráví o~svém zážitku, jak potkal Malého prince a~jedná zde i~jako autor.

Děj celé knihy začíná v~moment, kdy pilot havaruje v~poušti a~musí najít závadu 
letadla, aby se mohl vrátit zpět. Zde ovšem potká Malého prince, který se vrací
na místo, kde se poprvé na Zemi objevil. Malý princ žádá o~namalování beránka
a~ukazuje se dětská fantazie Malého prince.

Postupem času se oba stávají více blízcí a~Malý princ vypráví, jaké různé planety
navštívil a~jaké lidi tam potkal. Až na Zemi žil na každé planetě pouze jeden 
člověk~-- např.~král, pijan, zeměpisec, ekonom atd. 

Příběh končí \uv{smrtí} Malého prince a~jeho návratem domů.
Kniha je v~její podstatě jedna velká alegorie a~snaží se poukázat na některé aspekty
reálného života.
}
\parag{Téma a~motiv}{zamyšlení nad životem, pomíjivost, životní priority}
\parag{Časoprostor}{okolo roku vydání 1943, poušť}
\parag{Kompoziční výstavba}{chronologický děj, výrazné 3 části (seznámení s~Malým princem, 
vyprávění o~cestování po planetkách, konec a~odchod), rozděleno na kapitoly}
\parag{Literární druh a~žánr}{epika, próza, pohádka/povídka}

\newpart

\parag{Vypravěč}{ich forma, příběh vyprávěn z~pohledu ztroskotaného pilota i~malého prince }
\paragtable{Postavy}{
\textbf{Malý princ}&nesmělý, kreativní, hodně používá fantazii, dětská nevinnost\\
\textbf{pilot} & milý, přátelský, najde v~Malém princi své dětství\\
\textbf{květina} & velmi blízký přítel Malého prince, po čase pochopí své chování, sobeckost\\
\textbf{postavy na planetkách} & každá planetka, kterou navštíví Malý princ, má svoji specifickou postavu\\
}

\parag{Vyprávěcí způsob}
{nepřímá řeč s~velkým využitím řeči přímé}

\parag{Typy promluv}{dialogy; občasné a~málo časté monology}
%\parag{Veršová výstavba}{blankvers}

\newpart

\parag{Jazykové prostředky a~jejich funkce}	
{slova spisovná, občas hovorová či odborná slova}

\parag{Tropy a~figury a~jejich funkce}{metafory (stará opuštěná skořápka), přirovnání (jako padá strom), personifikace (květina zakašlala, 
studna zpívá), hyperbola (nekonečná poušť), eufemismus}
\parag{Kontext autorovy tvorby}{rok 1943, konec autorovy tvorby (1944 zahynul)}

\parag{Antoine de Saint-Exupéry}{
\begin{itemize}
\setlength\itemsep{0em}
\item 1900 -- 1944
\item vždy nadšený letec $\rightarrow$ vojenský letec
\item věnoval se i~psání a~hlavě znám jako spisovatel
\item Citadela, Noční let, Válečný pilot
\end{itemize}
}

\paragtable{Literární/obecně kulturní zasazení}{
\textbf{Erich Maria Remarque}&Na západní frontě klid\\
\textbf{James Joyce}&Odysseus\\
\textbf{Franz Kafka}&Proměny\\
}

}
