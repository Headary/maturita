%kat3
\bookname{Hobit}
\booksubname{aneb cesta tam a~zase zpátky}
\bookauthor{John Ronald Reuel Tolkien}

\bookcontent{

\newpart

\parag{Děj díla}{
K~hobitu Bilbovi, žijícím v~poklidné noře, přijde jednou čaroděj Gandalf
hledajíce účastníka pro dobrodružství, jehož cílem je získat zpět království
a~majetek trpaslíků od draka Šmaka. Bilbo prve odmítá, ovšem po příchodu všech
13 trpaslíků nakonec nabídku přijímá. Cestou narazí na mnoho zápletek
a~problému, od zajetí skřety, přes hru s~Glumem až po nalezení Prstenu moci. 

Na konci jejich snažení se dostanou k~Hoře, původnímu sídlu trpaslíku, kde
momentálně sídlí drak Šmak, který se zmocnil trpasličího bohatství. Draka se
podaří zabít s~pomocí občanů z~blízké vesnice a~začínají hádky o~příděl
znovunalezeného pokladu. 

Celý příběh končí tzn.~\emph{bitvou pěti armád}, kdy na jedné straně stojí
lidé, trpaslíci a~elfové a~na straně druhé skřeti a~vrrci. Bilbo prospává celý
boj a~nakonec se vrací s~Gandalfem domů se svým podílem zlata. Ovšem jeho dům,
hobití noru, dokázali tamní obyvatelé a~jeho příbuzní již z~velké části
rozprodat, a~proto se Bilbo uchyluje k~obnovování svého příbytku.
}
\parag{Téma a~motiv}{sebepoznání, boj dobra a~zla, odlišné pohledy na svět}
\parag{Časoprostor}{fiktivní země Středozemě, 2941 Třetího věku}
\parag{Kompoziční výstavba}{chronologická, podpříběhy}
\parag{Literární druh a~žánr}{epika; román, fantasie; próza s~poetickými písněmi}

\newpart

\parag{Vypravěč}{er forma}
\parag{Postavy}{
\begin{description}
\item[Bilbo Pytlík] hobit (rasa menší od lidí), přátelský, klidný, nepreferuje dobrodružství 
\item[Gandalf] vysoký kouzelník, moudrý, laskavý, libuje si v~kouření dýmky
\item[Trpaslíci] skupina 13 trpaslíku mířící do Hory, hlavní trpaslík Thorin Pavéza
\end{description}
}

\parag{Vyprávěcí způsob}
{spisovný, vymyšlené výrazy (vrrci, skřeti\dots), jazyky (sindarština,
quenijština) i~písmo (runy), vlastní geografie i~historie, básničky/písničky}

\parag{Typy promluv}{přímá (mezi postavami) i~nepřímá (vypravěč) řeč}
%\parag{Veršová výstavba}{blankvers}

\newpart

\parag{Jazykové prostředky a~jejich funkce}
{metafory, rýmované verše (Vál vítr přes uvadlý vřes, leč bez pohnutí stál tam
les\dots), spisovný jazyk, anafora archaizmy (svíce), gradace, pointa,
přirovnání (černé jako cylindry), elipsa}

\parag{Tropy a~figury a~jejich funkce}{archaizmy (tucet), metafora (zvlněné
hřebeny a~svahy klesající k~nížině), anafora (Bež hlašu pláče, běž nehtů štípá,
bež nohou škáře, bež pyšku pípá.)}

\parag{Kontext autorovy tvorby}{1. polovina 20. století, doba krátce před
druhou světovou válkou, základ moderní hrdinské fantasy}

\parag{John Ronald Reuel Tolkien}{
\begin{compactitem}
\item 3.~ledna~1892 -- 2.~září~1973
\item britský spisovatel, filolog a~univerzitní profesor
\item považován za otce moderní hrdinské fantasy
\item vytváří umělé země, kultury, jazyky, písma
\item další díla: Pán prstenů, Silmarilion, Příběhy Toma Bombadila
\end{compactitem}
}

\parag{Literární/obecně kulturní zasazení}{
\begin{description}
\item[C. S. Lewis] Letopisy Narnie
\item[J. M. Barrie] Petr Pan
\item[R. E. Howard] Povídky o~barbaru Conanovi
\end{description}
}
}
