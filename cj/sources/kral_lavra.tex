%kat2
\bookname{Král Lávra}
\booksubname{}
\bookauthor{Karel Havlíček Borovský}

\bookcontent{
\newpart

\parag{Děj díla}{
}

\parag{Téma a~motiv}{Kritika dokonalosti panovníků, monarchie, pohádkové motivy, král, upřímnost, laskavost, tajemství}
\parag{Časoprostor}{rok, místo}
\parag{Kompoziční výstavba}{chronologický děj, 3 hlavní části, retrospektivní části}
\parag{Literární druh a~žánr}{epiky, poezie, satirická báseň s~prvky pohádky}

\newpart

\parag{Vypravěč}{er forma, neosobní vypravěč}

\parag{Postavy}{
\begin{description}
	\item[Král Lávra] hlavní postava, panovník s~oslíma ušima, tajemný, hodný
	\item[Kukulín] kadeřník, vybrán k~ostříhání, dostal milost
	\item[Stará vdova] Kukulínova matka, milující
	\item[Červíček] basista, narážka na Čechy (šikovní muzikanti), jeho basa vyzrazuje královo tajemství
\end{description}
}

\parag{Vyprávěcí způsob}{
	přímá a~nepřímá řeč
}

\parag{Typy promluv}{pásmo vypravěče (lyrického subjektu), dialogy}
\parag{Veršová výstavba}{volný rým}

\newpart

\parag{Jazykové prostředky a~jejich funkce}{spisovný jazyk, jednoduchý jazyk, lidová mluva}

\parag{Tropy a~figury a~jejich funkce}{metonymie, personifikace, ironie, alegorie, přirovnání}
\parag{Kontext autorovy tvorby}
Národní obrození (19.~století), vzpoura proti německosti, revoluce 1848
(zrušení roboty), 1.~průmyslová revoluce; romantismus (18.--19.~století),
individualita a~cit

\parag{\getauthor}{
\begin{compactitem}
\item 1821 -- 1856
\item český básník, publicista, kritik, novinář (zakladatel Národních novin, redaktor Pražských novin)
\item politik~-- kritika absolutismus, církve, Rakouska-Uherska
\item národní obrozenec~-- proti vládě, zastánce Slovanů, rusofil
\item vyhnán do Brixenu (jižní Tyrolsko) -- Tyrolské elegie
\item národní obrození, realismus (obyčejný život~-- Neruda, Erben), romantismus (Mácha, Němcová), májovci
\item Epigramy, Tyrolské elegie, Obrazy z~Rus
\end{compactitem}
}

\parag{Literární/obecně kulturní zasazení}{
\begin{description}
	\item[Karel Jaromír Erben] Kytice
	\item[Jan Neruda] Povídky malostranské
	\item[Božena Němcová] Babička
\end{description}
}
}
