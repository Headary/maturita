%kat4
\bookname{Petrolejové lampy}
\booksubname{(Vyprahlé touhy)}
\bookauthor{Jaroslav Havlíček}

\bookcontent{

\newpart
\parag{Děj díla}
Příběh začíná narozením Štěpánky Kiliánové. Její dětství bylo propleteno
divadlem, sněním a nevelkým počtem přátel. Okolní lidé ji nemají většinou moc
v~lásce, z důvodu jejího vzhledu a jejího drsného chování. Stála si za svými
názory, které si uměla prosadit i přes společenskou nevhodnost. Žila v této
době v podstatě dvojí život: na jednu stranu trávila čas se svými bratranci na
statku, na stranu druhou byla nucena trávit čas s holčičkami z podobně
honosných rodin.

Během svého dospívání o Štěpánku nejeví zájem moc nápadníků z důvodu jejího
vzhledu. I přes její touhu mít muže a děti jí většina známostí moc dlouho
nevydrží. Ani zdánlivá známost s Pavlem Malinou nakonec nedopadá.

Pavel Malina se vydává studovat a následně se připojuje k armádě. Zde se z něj
stává líným a rozhazovačným. Pouze pije, kouří, a využívá velice rychlých
známostí. Rozhazuje všechny peníze, a tak je statek nucen platit všechny jeho
dluhy, čímž začíná chátrat. Nakonec se vrací zchátralý, zanedbaný a zničený.

Po návratu domů na statek se Pavel dozvídá, že zbytek domácnosti se chce zbavit
jejich největší přítěže, tedy jeho. Pavel však přichází s revolučním řešením,
jak statku navrátit jeho původní slávu. Jeho plánem se stane taktická svatba se
Štěpánkou Kiliánovou, která je velice dobře majetně zajištěna. Zbytku rodiny se
tento nápad líbil, a tak se Pavel vydal ke Kiliánům na námluvy. Přišel a
egoisticky oznámil Štěpánce svůj plán. Ta sice byla zaskočená, každopádně
vzhledem k touze po manželovi a k ne již nízkému věku nakonec na dohodu kývne.

Po svatbě ale Štěpánka naráží na krutou realitu. Pavel se vyhýbá jakémukoliv
intimnímu kontaktu, což Štěpánku deprimuje. Nakonec se dozvídá, že má pohlavně
přenositelnou chorobu. Tím se definitivně rozplývají všechny její sny o dětech.

Postupem času se Pavlům stav stále více zhoršuje a ze Štěpánky se stává pravá
selka starající se o statek. Situaci ani neusnadní smrt Štěpánčina otce, po
které se začne Pavlův fyzický i mentální stav rapidně zhoršovat, do doby kdy se
dočista nezblázní. Po mnoha událostech se Jan rozhodne odvézt bez vědomí
Štěpánky Pavla do ústavu. Štěpánka se snaží jej dostat zpět, ovšem kvůli jeho
stavu to již není možné. Několik měsíců na to Pavel umírá.

Příběh končí otevřeným koncem, kdy ovdovělá Štěpánka stále doufá v děti a Jana
napadá možnost to dát s ní dohromady. Nicméně výsledek tohoto uspořádání již
není jasný.

\parag{Téma a~motiv} Zničení životních snů a setkání s realitou; petrolejové
lampy, statek, vesnický život, majetek, sny, zklamání

\parag{Časoprostor} Jilemnice a Vejrychovsko; pozdní 2.~pol.~19.~stol.~a~přelom
19.~a~20.~stol.

\parag{Kompoziční výstavba} chronologický děj
\parag{Literární druh a~žánr} epika, próza, psychologický román

\newpart

\parag{Vypravěč}{er forma, vševědoucí}
\parag{Postavy}
\begin{description}
	\item[Štěpánka Kiliánová] hlavní postava, tlustá, nepřitažlivá, milá,
		snílek, zničená osudem
	\item[Pavel Malina] manžel Štěpánky, rozlítaný, nerozvážný, rozhazovačný,
		vypočítavý
	\item[Anna Kiliánová (Malinová)] matka Štěpánky, hodná, starostlivá, přeje
		si to nejlepší pro svoji dceru
	\item[Stavitel Kilián] otec Štěpánky, rázný avšak starostlivý, snaživý
	\item[Starý Malina] otec Jana a Pavla, bratr Anny, primární zájem o blaho
		jeho statku
	\item[Jan Malina] rozumný, rozvážný, zamilován do služebné
\end{description}

\parag{Vyprávěcí způsob}
spisovná mluva, alegorie, odborné termíny, personifikace, metafory,
slova se socialistickým zabarvením

\parag{Typy promluv} dialogy a vnitřní monology
%\parag{Veršová výstavba}

\newpart

\parag{Jazykové prostředky a~jejich funkce}
převážně spisovný, působivý, popisy prostředí

\parag{Tropy a~figury a~jejich funkce} metafora (\textit{Zmučená kostra se
táhla od stolu ke skříni.}), ironie (\textit{\uv{Hospodařit? Copak ty chceš
také hospodařit?}}), personifikace (\textit{Jaro se ohlašovalo.})

\parag{Kontext autorovy tvorby} 1.~polovina 20.~století, Československo, tvorba
za druhé světové války, meziválečná psychologická próza

\parag{Jaroslav Havlíček}
\begin{compactitem}
	\item 3.~února 1896~-- 7.~dubna 1943
	\item český spisovatel psychologické prózy
	\item narozen v Jilemnici, studium v Praze
	\item účastnil se bojů v Rusku, Itálii a na Slovensku
	\item smrt na zánět mozkových blan
	\item Neviditelný, Vlčí kůže
\end{compactitem}

\parag{Literární/obecně kulturní zasazení}
rozkvět meziválečné psychologické prózy z důvodu odvedení pozornosti od reality (doba okupace),
\begin{description}
	\item[Jarmila Glazarová] Vlčí jáma,
	\item[Vladislav Vančura] Rozmarné léto (imaginativní próza)
\end{description}
}
