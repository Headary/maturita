%kat4
\bookname{Smrt krásných srnců}
\booksubname{}
\bookauthor{Ota Pavel}

\bookcontent{
\vspace{-30pt}
\newpart

\parag{Děj díla}
\subparagraph{Nejdražší ve střední Evropě} V~první povídce tatínek koupil
rybník s~pouze jednou rybou. Panu Václavíkovi se pomstil tím, že mu prodal
ledničku bez jakéhokoliv vybavení. Když po něm chtěl pan Václavík vysvětlení,
tatínek mu řekl že to bylo jako s~tím rybníkem, navenek krásný ale uvnitř nic
nebylo.

\subparagraph{Ve službách Švédska} Druhá povídka ukazuje, jak byl tatínek
schopný obchodník ve firmě Elektrolux. V~prodeji vysavačů a~ledniček se stal
nejprve mistrem republiky, a~později i~mistrem světa. V~této povídce se
spřátelil s~věhlasným malířem Nechlebou. Chtěl od něj namalovat paní Irmu, do
které byl tajně zamilovaný, ale pan Nechleba odmítl.

\subparagraph{Smrt krásných srnců} Třetí povídka je o~tom jak tatínek jel svým
synům, kteří měli nastoupit do koncentračních táborů, pro pořádné poslední
jídlo před odjezdem a~přivezl jim jelena díky pomoci pana Proška a~Holana.

\subparagraph{Kapři pro Wehrmacht} Čtvrtá povídka vypráví o~odebrání tatínkova
milovaného rybníka Němci. Když se dozvěděl, že ho povolali do koncentračního
tábora, noc před odjezdem rybník vylovil. Po jeho odjezdu maminka s~mladým Otou
vyměňovali kapry za jídlo. Když nastal výlov rybníka, Němcům nezbyl ani jeden
kapr.

\subparagraph{Jak jsme se střetli s Vlky} Pátá povídka je o~soutěži v~chytání
ryb mezi rodinou Popperových a~rodinou Vlkových. Tehdy Vlkovi porazili Otu
a~tatínka.

\subparagraph{Otázka hmyzu vyřešena} Šestá povídka popisuje tatínkovy poválečné
obchody, kdy se mu už moc nedařilo. Začal s~prodejem mucholapek s~panem
Jehličkou, ale po neúspěchu tatínek podal výpověď a~pan Jehlička všechny
mucholapky spálil. Na konci se tatínek dozvěděl, že mucholapky byly hitem, když
je prodávali Holanďané pod názvem FlyKiller.

\subparagraph{Prase nebude!} Předposlední povídka vypráví o~práci rodičů v~JZD.
Za tu mají přislíbené celé prase. Přirozeně se těšili na zabíjačku, ovšem jim
bylo řečeno, že prase nebude. A~tak udělali jednoduchou věc a~prase si vzali
sami. Na konec jim ještě došel dopis, že jakožto bývalí zaměstnanci musí
zaplatit.

\subparagraph{Králíci s modrýma očima} V~poslední povídce Otův tatínek začal
prodávat králíky. Takhle to dělali asi 10 let, poté tatínek nechal králíky
tetovat a~vzal je na výstavu. Z~té se vrátil bez králíků, kteří měli
nedostatky, a~proto je vypustil na svobodu. Když přišel pěšky z~výstavy,
maminka mu rychle zavolala sanitku. Před odchodem dal ještě ceduli na vrátka:
PŘIJDU HNED, ale už se nikdy nevrátil.


\parag{Téma a~motiv}
vzpomínání na dětství za dob nacismu; dětství, příroda, válka, nacismus, totalita, obchod
\parag{Časoprostor}
období před, během a~po druhé světové válce; Buštěhrad, Křivoklátsko, Praha, Berounka
\parag{Kompoziční výstavba} % chronologická, retrospektivní...
8 jednotlivých povídek; chronologická~-- každý příběh chronologický, příběhy
seřazen v~čase; retrospektivní~-- vzpomínání na dětství
\parag{Literární druh a~žánr} % lyrika/epika, poezie/próza...
epika, próza, soubor autobiografických povídek

\vspace{-10pt}
\newpart
\vspace{-7pt}

\parag{Vypravěč} % er forma, neosobní vypravěč
ich forma, personální vypravěč
\parag{Postavy}
\begin{description}
\item[Ota Popper (Pavel)] vypravěč, sám autor
\item[Leo Popper (Pavel)] tatínek, hodný, starostlivý, pilný, energický, dobrý obchodník
\item[Herma Popper (Pavlová)] maminka, starostlivá, milující, hodná, tolerantní, laskavá
\item[Hugo, Jiří] bratři, v~příběhu moc nezmíněni
\item[Karel Prošek] strýček, pytlák, dobrý kamarád Oty, vztah k~přírodě
\item[Holan] vlčák, uloví Leovi srnce, věrný
\item[Irma a~ředitel] zaměstnávali Lea, do Irmy byl zamilován 
\end{description}

\parag{Vyprávěcí způsob} % řeč přímá, polopřímá, nevlastní přímá, nepřímá
přímá a~nepřímá řeč
\parag{Typy promluv} % monolog, dialog, pásmo vypravěče
dialogy, pásmo vypravěče
%\parag{Veršová výstavba}

\vspace{-10pt}
\newpart
\vspace{-7pt}

\parag{Jazykové prostředky a~jejich funkce} % spisovný/nespisovný jazyk, hovorový
spisovný jazyk, vulgarismy, hovorové výrazy, ironie, humor
\parag{Tropy a~figury a~jejich funkce} % metafora, metonymie, přirovnání, personifikace
metafory, přirovnání, personifikace

\parag{Kontext autorovy tvorby}
\begin{compactitem}
	\item 2. pol. 20. stol., 3. vlna válečné prózy
	\item vzpomínková próza, návraty, válka kulisou
	\item kritika režimu, cenzury\dots
	\item varování před totalitou a~potlačením lidskosti
	\item vězeňská literatura, historická próza
\end{compactitem}

\parag{\getauthor}
\begin{compactitem}
	\item 1930--1973
	\item rozen Otto Popper
	\item spisovatel, sportovní reportér, novinář
	\item představitel ztracené generace (2. světová válka)
	\item židovského původu, po transportu otce a~dvou starších bratrů žil jen s~matkou
	\item vztah ke sportu a~rybaření, pracoval se sportovní redakci Československého rozhlasu
	\item psychické problémy, několikrát v~psychiatrické léčebně
	\item zemřel na srdeční infarkt
	\item Jak jsem potkal ryby, Dukla mezi mrakodrapy
\end{compactitem}

\parag{Literární/obecně kulturní zasazení}
\begin{description}
	\item[Egon Bondy] Afghánistán, Nepovídka
	\item[Ladislav Fuks] Spalovač mrtvol
	\item[Arnošt Lustig] Modlitba pro Kateřinu Horovitzovou
\end{description}
}
