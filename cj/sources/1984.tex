%kat3
\bookname{1984}
\booksubname{}
\bookauthor{George Orwell}

\bookcontent{

\newpart

\parag{Děj díla}{
Děj se odehrává v~Anglii na Panské farmě. Zvířata nejsou spokojená
s~hospodařením pana Jonese a~poté, co jim staré prase Major vnukne nápad
revoluce, se obrátí proti člověku. Pan Jones se samozřejmě pokouší dostat farmu
zpět, ovšem neúspěšně.

Po revoluci nastává problém, kdo bude dá vést farmy. Vedení se ujímají dvě
prasata: Napoleon a~Kuliš. Nejdříve spolu vycházejí a~vytvářejí ideologii, že
všechna zvířata jsou si rovna a~že nikdy se nebudou chovat jako člověk. Zvířata
se starají o~farmu a~vede se jim poměrně dobře.

Postupem času Kuliš vytváří návrhy mlýnu, který by měl farmě usnadnit práci.
Napoleon ovšem svoji záští a~zrádcovstvím vyhání Kuliše z~farmy. Po čase si
i~přivlastní nápady mlýnu a~všechny Kulišovy činy zapírá. Má neskutečnou moc,
vychoval si smečku psů na ochranu a~přesvědčuje všechna zvířata, aby pracovala.

Takhle zvířata hospodaří několik let. Prasata se čím dál tím víc podobají
člověku, upravují si vlastní pravidla, která zavedla, povyšují se nad zvířata
až nakonec po velkém počtu podrazů a~přetvářky se spojují s~dalšími farmáři
a~již nejsou rovni další zvířatům a~v~podstatě se z~nich stává člověk.
}

\parag{Téma a~motiv}{komunizmus, totalismus, moc, zastrašování}
\parag{Časoprostor}{Panská farma (Farma zvířat), Anglie, 50.~léta 20.~století}
\parag{Kompoziční výstavba}{chronologický děj}
\parag{Literární druh a~žánr}{epika, próza, bajka/zvířecí alegorie}

\newpart

\parag{Vypravěč}{er forma}
\paragtable{Postavy}{
\textbf{Pan Jones} 	& původní majitel Panské farmy, který se zvířaty dobře nezacházel\\
\textbf{Major} 		& staré prase, jež jako první má myšlenku revoluce a~před svojí 
						smrtí podnítí všechna zvířata na farmě k~převratu; alegorie na Lenina\\
\textbf{Napoleon} 	& nejchytřejší prase a~zvíře vůbec, podvodník, pokrytec, zrádce;
						vytváří pravidla, která nedodržuje; představitel Stalina\\
\textbf{Kuliš} 		& další chytré prase, které se snaží o~vedení farmy; čestný, dobrý,
						nakonec vyhnán Napoleonem; vyobrazení Trockého.\\
\textbf{Pištík} 	& pravá ruka Napoleona, lže, poskakuje okolo Napoleona, přesvědčuje 
						zvířata, jak je farma dokonalá\\
\textbf{Boxer} 		& oddaný kůň, vždy pracovitý, věří v~ideologii\\
\textbf{Ovce} 		& hloupá zvířata, zmanipulovaná Napoleonem, stále pouze opakují 
						naučené fráze\\
\textbf{Psi} 		& Napoleonova tajná policie, zastrašují ostatní zvířata\\
}

\parag{Vyprávěcí způsob}{
spisovná mluva, alegorie, odborné termíny, personifikace, metafory,
slova se socialistickým zabarvením
}

\parag{Typy promluv}{přímá i~nepřímá řeč, dialogy i~monology}
%\parag{Veršová výstavba}{}

\newpart

\parag{Jazykové prostředky a~jejich funkce}{
odborné termíny (animalismus), metafory (ovce~-- nepřemýšlejí, jdou se stádem, psi~-- policie),
archaizmy (báchorky, senoseč)}

\parag{Tropy a~figury a~jejich funkce}{metafory, alegorie}
\parag{Kontext autorovy tvorby}{1. polovina 20. století, Anglie, ovlivněn světovými válkami 
a~následným režimem}

\parag{George Orwell}{
\begin{itemize}
\setlength\itemsep{0em}
\item 25.~června~1903 -- 21.~ledna~1950
\item britský novinář, spisovatel, esejista
\item inspirace a~popis totalitních ideologií
\item 1984, Barmské dny
\end{itemize}
}

\paragtable{Literární/obecně kulturní zasazení}{
\textbf{Anna Franková}&Deník Anny Frankové\\
\textbf{Isaac Asimov} &Já, robot\\
\textbf{Karel Čapek} &Rossumovi univerzální roboti\\
}
}
