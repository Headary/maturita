%kat3
\bookname{1984}
\booksubname{}
\bookauthor{George Orwell}

\bookcontent{

\newpart

\parag{Děj díla}
V první části díla je popsána ideologie a společnost na příkladu Winstona
Smithe. Společnost v Oceánii se dělí na Vnější Stranu, Vnitřní Stranu a
proletariát. Každý člen Vnější strany je zaměstnáván na některém
z ministerstev, která jsou různě pojmenována, ale ve skutečnosti všechna slouží
k tomu samému a to k měnění minulosti tak, aby korespondovala se současnou
stranickou ideologií, zjednodušování Newspeaku, či mučení. Veškeré emoce jsou
zakázané. Všude vládne udavačství.

I přes neustálé vymývání mozků se najdou ve společnosti lidé, kteří si
uvědomují, že takhle se žít nemá a že celá ideologie Strany je jen velká snůška
lží. Jedním z nich je právě i Winston. Nedokáže odolat a koupí si deník, kam si
zapisuje své protistranické myšlenky, přestože ví, že ho čeká jistá zkáza.

Ve druhé části díla se Winston se  zamiluje do úřednice Julie, s níž často
utíká mimo dosah špionážní techniky a pomáhá šíření revolučních myšlenek. Celou
dobu jsou však sledováni Charringtonem a nakonec zatčení Ideopolicií a mučeni
v celách Ministerstva Lásky. Je to tak příšerné, že si oba přejí umřít,
několikrát se navzájem zradí, přiznají i to, o čem nic neví, co nikdy neslyšeli
a o čem vědí, že není pravda. V okamžicích vrcholné hrůzy se přestanou navzájem
milovat.

Celá třetí část knihy popisuje právě ono mučení a ke konci se Winston i Julie
ocitají překvapivě na svobodě, ale již nejsou ti samí lidé, protože oba milují
Velkého bratra a necítí navzájem k sobě žádné city.

\parag{Téma a~motiv}{komunizmus, totalismus, moc, zastrašování}
\parag{Časoprostor}{rok 1984, Londýn -- fiktivní stát Oceánie}
\parag{Kompoziční výstavba}{chronologický děj, 3 hlavní části, retrospektivní části}
\parag{Literární druh a~žánr}{epika, próza, antiutopický román}

\newpart

\parag{Vypravěč}{er forma, neosobní vypravěč}

\parag{Postavy}
\begin{compactdesc}
	\item[Winston Smith] hlavní hrdina, člen vnější strany, nesouhlasí s ideologií
	\item[Julie] Winstonova přítelkyně, na první pohled členka strany, ve skutečnosti proti režimu
	\item[O’Brian] tváří se jako nepřítel strany, ale je věrný následovník, vyděrač
	\item [Goldstein]  nepřítel strany, hlavní postava odbojového hnutí
	\item[Velký Bratr] vůdce strany, má vždy pravdu, pravděpodobně fiktivní postava
	\item[Syme, Parsons] vaporizovaní lidé z Winstonova okolí
\end{compactdesc}


\parag{Vyprávěcí způsob}
přímá, nepřímá, nevlastní přímá, nepřímá řeč


\parag{Typy promluv}{monology a převážně dialogy, pásmo vypravěče}
%\parag{Veršová výstavba}{}

\newpart

\parag{Jazykové prostředky a~jejich funkce}{
spisovný jazyk, vlastní řeč, odborné termíny (valorizace), vlastní slova
(angsoc, ideozločin), slogany, řečnické otázky}

\parag{Tropy a~figury a~jejich funkce}{metafory, archaizmy, personifikace}
\parag{Kontext autorovy tvorby}{1.~polovina 20.~století, Anglie, ovlivněn
světovými válkami a~následným režimem}

\parag{George Orwell}{
\begin{compactitem}
\item 25.~června~1903 -- 21.~ledna~1950
\item vlastním jménem Eric Arthur Blair
\item britský novinář, spisovatel, esejista
\item poválečná literatura, nepodléhá poválečné pozitivitě
\item narozen v Indii
\item práce pro imperialistickou policii $\rightarrow$ nenávist
\item Španělská občanská válka $\rightarrow$ na straně marxistů proti fašistům $\rightarrow$ proti jakémukoliv totalitnímu režimu
\item inspirace a~popis totalitních ideologií
\item Farma zvířat, Barmské dny
\end{compactitem}
}

\parag{Literární/obecně kulturní zasazení}{
\begin{compactdesc}
\item[Isaac Asimov] Já, robot
\item[Ray Bradbury] $451^\circ$ Fahrenheita
\item[Karel Čapek] Rossumovi univerzální roboti
\end{compactdesc}
}
}
