%kat2
\bookname{Tyrolské elegie}
\booksubname{}
\bookauthor{Karel Havlíček Borovský}

\bookcontent{
\newpart

\parag{Děj díla}
Borovský se svěřuje měsíčku a ptá se ho, jak se mu v Brixenu líbí. Ať se ho nebojí,
že není zdejší. Následně mu popisuje své zatčení a cestu do vyhnanství, od
příchodu policistů -- v nočních hodinách, smutné loučení s rodinou, rodným
městem a vlastí. Vlastně ani nevěděl, kam jede a jestli se vůbec někdy vrátí.
To vše kvůli tomu, že klade velký odpor vůči Bachovu absolutismu a prosazuje
svobodu lidí a projevu. Proto se Bachovi nehodí a Borovský výsměšně popisuje,
že mu \uv{doktor Bach naordinoval čerstvý vzduch v Alpách s okamžitým
odjezdem}. Na cestě ho provázel policejní komisař Dedera a další strážní
(žandarmové). Vrcholem skladby je příhoda se splašenými koňmi po cestě přes
Alpy (8. zpěv) -- kočár se řítí po příkré cestě, všichni policisté z něj
vyskákali, když jim zajatec připomenul příběh z bible o Jonášovi -- jen
\uv{hříšník} Borovský měl čisté svědomí a v klidu dojel k poště, kde stačil
povečeřet, než pochroumaní strážci dokulhali za ním. Satiricky celou situaci
komentoval a neodpustil si ostrou kritiku režimu, který chce na šňůrce vodit
celé národy, ale neukočíruje ani čtyři koně\dots Nakonec všichni dojeli do
Brixenu a Havlíček zůstává pod dohledem místních orgánů až do roku~1855.

\parag{Téma a~motiv}
Kritika vlády, policie, Bachův absolutismus -- nesvoboda, Havlíčkovy pocity;
ironie, noc, osud
\parag{Časoprostor}
cesta z domu (Borová) do vyhnanství v Brixenu
\parag{Kompoziční výstavba} % chronologická, retrospektivní...
chronologická; 9 zpěvů -- každý popisuje část cesty
\parag{Literární druh a~žánr} % lyrika/epika, poezie/próza...
lyriko-epika, poezie, satirická a kritická báseň (elegie = žalozpěv, parodie na
elegii -- humor, ironie\dots)
\newpart

\parag{Vypravěč}
ich forma, personální vypravěč
\parag{Postavy}
\begin{compactdesc}
\item[Karel Havlíček Borovský] vypravěč, ironický, odvážný, výsměšný, rezignovaný
\item[pan Bach] tehdejší ministr Rakouska-Uherska (Bachův absolutismus),
	politik, \uv{padouch}
\end{compactdesc}

\parag{Vyprávěcí způsob} % řeč přímá, polopřímá, nevlastní přímá, nepřímá
přímá řeč, nepřímá řeč
\parag{Typy promluv} % monolog, dialog, pásmo vypravěče
monology, dialogy, pásmo vypravěče
\parag{Veršová výstavba}
přerývaný rým (ABCB), střídavý (ABAB), trochej

\newpart

\parag{Jazykové prostředky a~jejich funkce} % spisovný/nespisovný jazyk, hovorový
obecná řeč, cizí názvy, prostý text, občasná nespisovná mluva
\parag{Tropy a~figury a~jejich funkce} % metafora, metonymie, přirovnání, personifikace
ironie, satira, personifikace, apostrofa (oslovení neživé věci), metafora (já
jsem z kraje muzikantů), epiteton (hořká chvilka, tajná svědek)

\parag{Kontext autorovy tvorby}
\begin{compactitem}
	\item národní obrození (19.~století)
	\item vzpoura proti německosti
	\item revoluce 1848 (zrušení roboty), 1.~průmyslová revoluce
	\item romantismus (18.--19.~století) -- individualita a cit
\end{compactitem}

\parag{\getauthor}
\begin{compactitem}
\item 1821 -- 1856
\item český básník, publicista, kritik, novinář (zakladatel Národních novin,
	redaktor Pražských novin)
\item politik -- kritika absolutismus, církve, Rakouska-Uherska
\item národní obrozenec -- proti vládě, zastánce Slovanů, rusofil
\item vyhnán do Brixenu (jižní Tyrolsko) -- Tyrolské elegie
\item národní obrození, realismus (obyčejný život -- Neruda, Erben),
	romantismus (Mácha, Němcová), májovci
\item Epigramy, Tyrolské elegie, Obrazy z Rus
\end{compactitem}

\parag{Literární/obecně kulturní zasazení}
\begin{compactdesc}
	\item[Karel Jaromír Erben] Kytice
	\item[Jan Neruda] Povídky malostranské
	\item[Božena Němcová] Babička
\end{compactdesc}
}
