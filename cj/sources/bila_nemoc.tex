%kat4
\bookname{Bílá nemoc}
\booksubname{}
\bookauthor{Karel Čapek}

\bookcontent{
\newpart

\parag{Děj díla}
V~jisté, blíže neurčené zemi, se rozmohla epidemie tzv.~bílé nemoci. Jde
pravděpodobně o~jakousi formu malomocenství, která se projevuje bílými skvrnami
na kůži. Šíří se dotykem a~ze začátku pouze mezi chudými a~starými lidmi.

Zemi ovládá diktátor Maršál, který se připravuje na výbojnou válku proti
menšímu sousednímu státu. V~tom mu pomáhá vlastník zbrojovek baron Krüg.
Mladému doktorovi Galénovi se mezitím podaří najít lék proti bílé nemoci, ale
léčí jen chudé lidi, kteří nemají moc ani vliv. Těm vlivnějším Galén lék vydat
odmítá.

Když onemocní baron Krüg, zavolá Galéna, aby ho vyléčil. Ten mu dává podmínku
-- musí okamžitě zastavit zbrojení, jinak mu lék nebude vydán. Krüg žádá
Maršála, aby uzavřel mír, ten však nehodlá se zbrojením přestat a~Krüg ze
zoufalství spáchá sebevraždu. Po jeho smrti začíná útok na sousední zemi. Avšak
také Maršál na sobě zpozoruje bílou skvrnu. Maršálova dcera pozve Galéna, aby
Maršála vyléčil, ten má ale stále stejnou podmínku~-- Maršál musí uzavřít mír.

Teprve na naléhání své dcery a~mladého Krüga, je ochoten uzavřít mír a~splnit
tak Galénovu podmínku pro poskytnutí léku. Dá zavolat Galéna, ale ten se před
Maršálovým palácem střetne se skandujícím davem, a~protože se snaží odporovat
jejich provolávání hesla:\uv{Ať žije Maršál, ať žije válka!} je on i~lék davem
ušlapán. Dav v~tu chvíli netuší, že tak vlastně zabili oslavovaného vůdce.
Navíc na závěr vypuká válka.

\parag{Téma a~motiv}
epidemie neznámé nemoci, její následky a~snaha přestat válčit, alegorie na
nacistickou moc; nemoc, válka, lék, moc
\parag{Časoprostor}
fiktivní čas i~prostor, předloha~-- Evropa 20.~století
\parag{Kompoziční výstavba} % chronologická, retrospektivní...
3 jednání (5 obrazů, 6 obrazů a~3 obrazy), chronologická výstavba
\parag{Literární druh a~žánr} % lyrika/epika, poezie/próza...
drama, prvky tragédie, próza

\newpart

\parag{Vypravěč} % er forma, neosobní vypravěč
chybí vypravěč

\parag{Postavy}
\begin{description}
	\item[doktor Galén] původem Řek, zásadový, chytrý, skromný, laskavý,
		roztržitý, pacifista, snaha ho konci války
	\item[Maršál] obraz Hitlera, diktátor, krutý, zásadový, tvrdý, nakonec
		povolí, ale již pozdě
	\item[doktor Sigelius] dvorní rada, vede kliniku, snaží se dostat z~Galéna
		recept na lék, panovačný, nekompromisní, skeptický, povýšený
	\item[baron Krüg] vlastník továrny na zbraně, přítel Maršála, jeho umírněná
		verze
\end{description}

\parag{Vyprávěcí způsob} % řeč přímá, polopřímá, nevlastní přímá, nepřímá
přímá řeč
\parag{Typy promluv} % monolog, dialog, pásmo vypravěče
dialogy~-- repliky, scénické poznámky
%\parag{Veršová výstavba}

\newpart

\parag{Jazykové prostředky a~jejich funkce} % spisovný/nespisovný jazyk, hovorový
spisovný jazyk, cizí názvy (latina, angličtina, němčina, francouzština)
\parag{Tropy a~figury a~jejich funkce} % metafora, metonymie, přirovnání, personifikace
aposiopese~-- přerušení mluveného slova, elipsy, přirovnání, personifikace

\parag{Kontext autorovy tvorby}
\begin{compactitem}
	\item jedno z~prvních děl
	\item meziválečná česká próza
	\item humanistická demokratická literatura
	\item civilismus; inspirace novou technikou
	\item pragmatismus
	\item vrstevníci Josef Čapek, Karel Poláček, Eduard Bass
\end{compactitem}

\parag{\getauthor}
\begin{compactitem}
	\item 1890--1938
	\item spisovatel, novinář, politik
	\item spolupráce s~bratrem Josefem
	\item poznamenán 1.~sv.~válkou, nedůvěra k~nové technice (využití na zabíjení)
	\item okouzlen civilismem~-- přemýšlení o~teoretickém budoucím neovládnutí
		techniky
	\item tvorba mezi světovými válkami
	\item Bílá nemoc, Ze života hmyzu, Válka s~mloky
\end{compactitem}

\parag{Literární/obecně kulturní zasazení}
\begin{description}
	\item[Josef Čapek] Básně z~koncentračního tábora
	\item[Karel Poláček] Bylo nás pět
	\item[Eduard Bass] Cirkus Humberto
	\item[Jan Werich] Fimfárum
\end{description}
}
