%kat4
\bookname{Vyšetřování ztráty třídní knihy}
\booksubname{}
\bookauthor{Ladislav Smoljak}

\bookcontent{

\newpart

\parag{Děj díla}{
Třídní učitel vstoupí do třídy a~snaží se klukovskou třídu 8.C, zastoupenou
obecenstvem, přesvědčit, aby vrátili třídní knihu, která se před sedmi lety 
ztratila. Snaží se na žáky jít mile, přátelsky, ovšem poté přidává i~výhružky.
Čte žákům oběžník pana ředitele a~má snahu přesvědčit žáky, aby nebyli 
vyslýcháni i~jím.

Pan ředitel ze začátku netuší, jakého činu se třída dopustila. Po zjištění
informace od pana učitele přemlouvá žáky, aby vrátili knihu, než dojde
inspektor. Má zjevnou obavu z~jeho příchodu.

Inspektor vchází do třídy s~velkou hudební skříní zavěšenou kolem krku
a~více než ztracená kniha ho trápí ukradené čepice v~jiné škole.
Pan učitel s~panem ředitel opakují rozhovor před inspektorem,
marně.

Jako poslední se na scéně objevuje zemský školní rada. Ten vrhá strach na všechny
tři již dříve přítomné pouhou svojí přítomnosti. Moc tomu nepomáhá fakt, že za celou
dobu nepromluví jediné slovo. Inspektor, pan ředitel i~pan učitel provádí znovu
celý výslech třídy a~zamaskovávají ztrátu třídní knihy. Nakonec si zemský
školní rada vytáhne svačinu, nabídne ostatním na jevišti a~pan ředitel vybízí
žáky, aby se také pustili do svých svačin.

}
\parag{Téma a~motiv}{třídní kniha, chování mládeže; třída, škola}
\parag{Časoprostor}{škola, 8. třída, doba Rakousko-Uherska}
\parag{Kompoziční výstavba}{chronologická}
\parag{Literární druh a~žánr}{drama; komedie}

\newpart

\parag{Vypravěč}{ich forma}
\paragtable{Postavy}{
\textbf{Učitel} & představuje klasického učitele, vlídný i~zastrašující, obává se ředitele, inspektora i~zemského školního rady\\
\textbf{Ředitel} & vlídný, příjemný, snaží se vyjednávat, bojí se inspektora i~zemského školního rady\\
\textbf{Inspektor} & zaměřený na svoji práci, milý, starší, bojí se rady\\
\textbf{Zemský školní rada} & nadřazený, nemluví, nahání strach pouze funkcí\\
\textbf{Děti ve třídě} & postava neexistující fyzicky, zastoupena publikem, nikdy nemluví\\
}

\parag{Vyprávěcí způsob}
{přímá řeč, nadávky (hajzl, debil, blbeček), zdrobněliny(docela malilinko, trošičku)}

\parag{Typy promluv}{monolog s~publikem, později dialogy, scénické poznámky}
%\parag{Veršová výstavba}{blankvers}

\newpart

\parag{Jazykové prostředky a~jejich funkce}
{anakolut (Ale máme tady jednu takovou nepěknou, ošklivý příklad.), vulgarismy}

\parag{Tropy a~figury a~jejich funkce}{
přirovnání (Ale po třídní knize jako když se zem slehne.), synekdocha (A~jsme 
na lopatkách.), ironie (A~to je mi pěkné!)
}

\parag{Kontext autorovy tvorby}{druhé dílo od fiktivní postavy Járy Cimrmana, 70.~léta 20.~století}

\parag{Ladislav Smoljak}{
\begin{itemize}
\setlength\itemsep{0em}
\item 9.~prosince~1931 -- 6.~června~2010
\item režisér a~scénárista filmový i~divadelní, herec
\item vystudoval matematiku a~fyziku na vysoké škole pedagogické
\item založil a~vedl společně se Z.~Svěrákem a~J.~Šebánkem Divadlo Járy Cimrmana
\item autor mnoha děl od Járy Cimrmana
\end{itemize}
}

\paragtabletext{Literární/obecně kulturní zasazení}{
\begin{itemize}
\setlength\itemsep{0em}
\item uvolněnější doba komunismu, před Pražským jarem (1968)
\item mystifikační práce jakožto protesty vůči režimu, ale stále přijatelné cenzurou
\end{itemize}
}{
\textbf{Jan Skácel} & Kolik příležitostí má růže \\
\textbf{Karel Ptáčník} & Noc odchází ráno \\
\textbf{Bohumil Hrabal} & Ostře sledované vlaky \\
\textbf{Václav Havel} & Vyrozumění
}






}
