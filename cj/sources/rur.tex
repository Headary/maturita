%kat4
\bookname{R.U.R.}
\booksubname{}
\bookauthor{Karel Čapek}

\bookcontent{
\newpart

\parag{Děj díla}
Za Dominem přijede Helena, dcera prezidenta, a~chce vidět závod na výrobu
robotů. Přijela sem, aby promluvila k robotům, proč si nechají líbit zacházení,
jako by byli stroje. Roboti jsou nerozeznatelní od lidí, ale nemají city
a~lidské funkce. Proto dojde k záměně, kdy si nejprve Helena myslí o~asistentce
robotce Sulle, že je žena a~následně o~skutečných lidech je přesvědčena, že
jsou roboti a~tedy k nim i~promlouvá jako k robotům.

Helena v továrně zůstane a~vezme si Domina a~děj se přesune o~deset let
později.

Po deseti letech se roboti začínají používat i~jako vojáci do válek. Helenu
trápí, že nemají vlastní rozum a~city, a~proto přemluví doktora Galla, aby se
pokusil takové roboty vyrobit. Svět začíná být plný robotů, lidem se přestávají
rodit děti a~roboti se začínají proti lidem bouřit.

Helena už nechce, aby se roboti vyráběli, a~tak spálí plány na jejich výrobu.
Mezitím vedoucí fabriky už ví o~vzbouření a~v den, kdy je to přesně deset let,
co Helena přijela na Rossumový ostrov, jí věnují dělovou loď. Bohužel už nikdo
nestihne odjet, roboti přicházejí na ostrov, nabírají vše a~zabijí všechny lidi
kromě Alquista, který jediný pracuje jako robot.

Po vyvraždění lidstva si roboti uvědomí, že jejich životnost je omezená a~bez
plánů na výrobu za dvacet let vyhynou. Prosí teda Alquista, aby znova našel
plány na jejich výrobu. Alquistovi se to však nepodaří. Najde dva roboty Prima
a~Helenu, kteří v sobě mají city, lásku a~jsou jiní než ostatní. Možná že právě
tito dva jsou počátkem nové civilizace.


\parag{Téma a~motiv}
budoucnost plná techniky a~její vliv na lidstvo; roboti, technologie, lidstvo,
porodnost, budoucnost
\parag{Časoprostor}
neurčitá budoucnost, Rossanův ostrov~-- ostrov na výrobu robotů.
\parag{Kompoziční výstavba} % chronologická, retrospektivní...
předehra a~3 dějství; chronologická výstavba
\parag{Literární druh a~žánr} % lyrika/epika, poezie/próza...
vědecko-fantastické a~satiricko-kritické drama, činohra (divadelní hra)

\newpart

\parag{Vypravěč} % er forma, neosobní vypravěč
chybí vypravěč
\parag{Postavy}
\begin{description}
\item[Harry Domin] z~latiny \uv{pán}, ředitel továrny, snaha vytvořit nové
	bytosti, vymknutí prvotní myšlenky mimo kontrolu
\item[Helena] manželka Dominova, ztělěsnění ženského, citového, laskavého
	přístupu k~robotům i~lidem, krásná, zástupce žen, neplodná, zbytečná
\item[Alquist] stavitel robotů, zástupce pracujícího lidu, v~lásce práce rukami
\item[Dr. Gall] ředitel fyziologického ústavu robotů, lékař robotů (paralela
	s~dr. Gallénem~-- Bílá nemoc)
\item[Fabry] generální technický ředitel, největším zájmem pokrok
\item[Busman] komerční ředitel, businessman, snaží se roboty vyplatit
\item[roboti Primus a~Helena] první roboti projevující lásko, jako Adam a~Eva
\end{description}

\parag{Vyprávěcí způsob} % řeč přímá, polopřímá, nevlastní přímá, nepřímá
přímá řeč~-- repliky
\parag{Typy promluv} % monolog, dialog, pásmo vypravěče
dialogy
%\parag{Veršová výstavba}

\newpart

\parag{Jazykové prostředky a~jejich funkce} % spisovný/nespisovný jazyk, hovorový
spisovný jazyk, neologismy (slovo \emph{robot}), termíny
\parag{Tropy a~figury a~jejich funkce} % metafora, metonymie, přirovnání, personifikace
personifikace, apostrofa (\uv{\emph{Změklé a~zmodralé rty, co to breptáte?}}), eufemismus (\uv{\emph{nebožtík}})

\parag{Kontext autorovy tvorby}
\begin{compactitem}
	\item jedno z~prvních děl
	\item meziválečná česká próza
	\item humanistická demokratická literatura
	\item civilismus; inspirace novou technikou
	\item pragmatismus
	\item vrstevníci Josef Čapek, Karel Poláček, Eduard Bass
\end{compactitem}

\parag{\getauthor}
\begin{compactitem}
	\item 1890--1938
	\item spisovatel, novinář, politik
	\item spolupráce s~bratrem Josefem
	\item poznamenán 1.~sv.~válkou, nedůvěra k~nové technice (využití na zabíjení)
	\item okouzlen civilismem~-- přemýšlení o~teoretickém budoucím neovládnutí
		techniky
	\item tvorba mezi světovými válkami
	\item Bílá nemoc, Ze života hmyzu, Válka s~mloky
\end{compactitem}

\parag{Literární/obecně kulturní zasazení}
\begin{description}
	\item[Josef Čapek] Básně z~koncentračního tábora
	\item[Karel Poláček] Bylo nás pět
	\item[Eduard Bass] Cirkus Humberto
	\item[Jan Werich] Fimfárum
\end{description}
}
