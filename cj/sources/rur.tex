%kat4
\bookname{R.U.R.}
\booksubname{}
\bookauthor{Karel Čapek}

\bookcontent{
\newpart

\parag{Děj díla}
\parag{Téma a~motiv}
\parag{Časoprostor}
\parag{Kompoziční výstavba} % chronologická, retrospektivní...
\parag{Literární druh a~žánr} % lyrika/epika, poezie/próza...

\newpart

\parag{Vypravěč} % er forma, neosobní vypravěč
\parag{Postavy}
\begin{description}
	\item
\end{description}

\parag{Vyprávěcí způsob} % řeč přímá, polopřímá, nevlastní přímá, nepřímá
\parag{Typy promluv} % monolog, dialog, pásmo vypravěče
%\parag{Veršová výstavba}

\newpart

\parag{Jazykové prostředky a~jejich funkce} % spisovný/nespisovný jazyk, hovorový
\parag{Tropy a~figury a~jejich funkce} % metafora, metonymie, přirovnání, personifikace

\parag{Kontext autorovy tvorby}
\begin{compactitem}
	\item
\end{compactitem}

\parag{\getauthor}
\begin{compactitem}
	\item
\end{compactitem}

\parag{Literární/obecně kulturní zasazení}
\begin{description}
	\item
\end{description}
}
