%kat3
\bookname{451 stupňů Fahrenheita}
\booksubname{}
\bookauthor{Ray Bradbury}

\bookcontent{
\newpart

\parag{Děj díla}
Děj se odehrává ve velmi hektické budoucnosti, kdy lidé neustále někam
spěchají, starají se pouze o~vlastní potěšení, nezajímá je to, co se děje
v~jejich okolí. Zásadně nečtou, knihy jsou považovány za zlo (protože můžou
urazit city menšin a~ty tak tlačí na to, aby vše bylo co nejabstraktnější),
které musí být zničeno.

Hlavním hrdinou je požárník Guy Montag. Domy se vyrábějí z~ohnivzdorných
materiálů a~požárníci už nemají co na práci, vyhledávají tedy nelegální
přechovavatele knih a~vypalují jejich knihovny, často i~celé domy. Požárníci
mají ve znaku číslo 451.

Montag svou práci miluje, ale jeho nová sousedka Clarissa McClellanová, dívka
pocházející z~rodiny podivínů (vzpomínajících na minulost) ho svými řečmi
nahlodá a~Guy začne přemýšlet o~tom, co se v~knihách asi tak píše. Při jedné
akci Guy ukradne z~domu nelegálních přechovavatelů několik knih, vezme je domů
a~v~práci nahlásí, že je nemocný. Velitel Beatty si vše domyslí a~přijde
k~Montagovi domů a~vypoví mu minulost hasičů. Také řekne, že když hasič ukradne
knihu a~do jednoho dne jí spálí, nic se mu nestane. Po odchodu Beattyho se
Montag přizná své ženě, že ve ventilátoru schovává další knížky (našla tu jednu
pod polštářem). Ona začne šílet a~chce je spálit, ale Montag ji přemluví, že se
na ně alespoň podívají.

Guy ho sice poslechne, ale nechá spálit pouze jedinou knihu na radu profesora
Fabera, kterého kdysi potkal v~parku. Beatty se však doví, že Guyovi doma ještě
zbývají knihy, a~na dalším výjezdu hasičů zastaví před Montagovým domem.

Hrdinu udala jeho žena Mildred a~její přítelkyně, jimž předtím četl básničky
právě z~ukradené knihy. Guy proto zapaluje svůj dům a~všechno v~něm. Avšak
poté, co Beatty přijde na jeho tajné spojení s~Faberem, je ho Guy nucen zabít
dát se na útěk. Policie okamžitě vysílá mechanického Ohaře, aby Montaga
vystopoval, ale Guy se v~profesorově domě převléká do Faberových šatů, které
navíc polije alkoholem, a~daří se mu Ohařovi utéct korytem řeky. Celá honička
ovšem běží naživo v~televizi a~policie si nemůže dovolit neuspět. Pošle tedy
Ohaře k~místu, kde se každý den prochází nějaký člověk a~nechá zabít nevinného
muže, aby oklamali bavící se lidi.

Guy se mezitím dostává do lesa ke skupině bývalých vysokoškolských profesorů,
kteří putují z~místa na místo a~pamatují si vědomosti z~knih (každý si pamatuje
nějakou). Montag, jelikož sám nějaké knihy četl, se k~nim přidává.

Po nějaké době propukne válečný konflikt a~město plné \uv{benzínových štvanců}
a~věčně se bavících lidí je během několika minut zničeno bombardováním.
Vzdělanci se vydávají městu na pomoci tím, co mají~-- vědomostmi získanými
z~knih.


\parag{Téma a~motiv}
povrchnost a~lehká manipulace lidmi, varování před budoucností technologie, budoucnost, technologie, zaslepenost, hloupost
\parag{Časoprostor}
neznámá budoucnost; místo nespecifikováno, nejspíše USA
\parag{Kompoziční výstavba} % chronologická, retrospektivní...
3 části; chronologická výstavba, retrospektivní prvky
\parag{Literární druh a~žánr} % lyrika/epika, poezie/próza...
epika, próza, román~-- sci-fi, antiutopie
\newpart

\parag{Vypravěč} % er forma, neosobní vypravěč
er forma, neosobní vypravěč
\parag{Postavy}
\begin{description}
	\item [Guy Montag] požárník, touží po poznání, statečný, horlivý, netrpělivý, oddaný práci
	\item [Mildred] manželka Montaga, povrchní, prázdná, nezájem k~ničemu kromě televize (propagandy), model poslušného občana
	\item [Clarissa McClellan] mladá dívka, podivín, starý pohled na svět, otevře Montagovy oči
	\item [Velitel Beatty] nadřízený, prohnaný, oddaný svému povolání
	\item [Profesor Faber] intelektuál, starý, studovaný, poučuje Montaga
\end{description}

\parag{Vyprávěcí způsob} % řeč přímá, polopřímá, nevlastní přímá, nepřímá
přímá řeč
\parag{Typy promluv} % monolog, dialog, pásmo vypravěče
dialogy, pásmo vypravěče
%\parag{Veršová výstavba}

\newpart

\parag{Jazykové prostředky a~jejich funkce} % spisovný/nespisovný jazyk, hovorový
spisovný a~bohatý jazyk
\parag{Tropy a~figury a~jejich funkce} % metafora, metonymie, přirovnání, personifikace
metafora, metonymie, přirovnání, personifikace

\parag{Kontext autorovy tvorby}
\begin{compactitem}
	\item období po~2.~světové válce
	\item rozvoj techniky $\rightarrow$ inspirace do sci-fi
	\begin{compactitem}
		\item studená válka
		\item jaderné technologie
		\item komunikační technologie
		\item kosmické závody
	\end{compactitem}
	\item 60.~léta~-- hnutí hippies
	\item válečná literatura, existencialismus
	\item beat generation (USA), rozhněvaní mladí muži (UK)
	\item antiutopie, sci-fi
	\item magický realismus, postmoderna, absurdní drama
\end{compactitem}

\parag{\getauthor}
\begin{compactitem}
	\item 1920--2013
	\item americký spisovatel, esejista, scénarista
	\item matka Švédka $\rightarrow$ zájem o~skandinávskou a~řeckou mytologii
	\item spisovatel sci-fi, fantasy a~hororu
	\item inspirace: divadlo, film, kouzla, komiksy
	\item Marťanské povídky, Ilustrovaná žena, Slunce a~stín
\end{compactitem}

\parag{Literární/obecně kulturní zasazení}
\begin{description}
	\item[George Orwell] 1984, Farma zvířat
	\item[Issac Asimov] Já, robot
	\item[Arthur C. Clarke] Vesmírná odysea 2001
\end{description}
}
