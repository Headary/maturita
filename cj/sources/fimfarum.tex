%kat4
\bookname{Fimfárum}
\booksubname{}
\bookauthor{Jan Werich}

\bookcontent{
\newpart

\parag{Děj díla}{
Dílo se skládá z~19 jednotlivých pohádek, každou se svým vlastním příběhem
a~postavami. Najdeme zde vysmívání se lidským nešvarům či vystupování
tradičních nadpřirozených bytostí, králů a~princezen a~reálií moderního světa
(automobily, koloběžka, telefon, protialkoholní léčebna).

\subparagraph{Královna Koloběžka První}
	Chytrá dcera mlynáře Zdenička, která se dostane svojí chytrostí (a~tím
	přízní krále) na post manželky a~následně i~královny. A~protože dojela na
	hrad poprvé na koloběžce, je pojmenovaná Královna Koloběžka.

\subparagraph{Fimfárum}
	Manželka se chce zbavit kováře, aby mohla mít poměr s~lokajem. Kovář
	dostane od lokaje postupně 3~úkoly~-- ukovat řetěz, přemístit řeku a~sehnat
	fimfárum~-- jinak bude oběšen. Čert a~Vodník mu pomohou a~nakonec
	i~fimfárum dostane, které zkamení celou vesnici.

\subparagraph{O~rybáři a~jeho ženě}
	Rybář najde rybku, která plní přání. Žena si přeje čím dál tím větší
	bohatství, až nakonec skončí zpět ve svém původním obydlí~-- v~láhvi od
	octa.

\subparagraph{Jak na Šumavě obři vyhynuli}
	Jednou na Šumavě žili obři, kteří soutěžili o~to, kdo je nejvyšší. A~protože
	chtěli vyhrát, tak podváděli a~když na to došli, tak se spustila velká bitka.
	Dlouho po skončení trvalo, než se lidé vrátili zpět.

\subparagraph{Až opadá listí z~dubu}
	Sedlák Čupera hodně pil, neustále byl opilý. Jednou na mýtině k~němu přišel
	čert, že mu pomůže, ale když mu dá to, co neví, že má. Čupera souhlasil a~doma
	zjistil, že se mu narodil syn. Za 5~let šel Čupera zas na mýtinu, kde převezl
	čerta, ale zas zanedlouho šel na mýtinu a~poprosil ho, jestli by ho neodnaučil
	pít. Vzal ho do pekla na dlouhou odvykací kúru, kde do něj nalili litry
	kořalky, až se mu to pití vysloveně znechutilo. Na to si upsal svoji duši, pro
	kterou si měl čert přijít, až opadá listí z~dubu. Jednou po roce na okno
	zaťukal čert a~Čupera ho v~klídku odvedl k~dubu a~vysvětlil mu, že listí z~dubu
	nikdy neopadá.

\subparagraph{Lakomá Barka}
	Barka byla hodně lakomá pomocnice faráře. Jednou jim učitel z~nouze ukradl
	jedno prase ze sklepa, protože měl moc dětí a~nevěděl, jak je uživit. Když
	se to Barka dozvěděla, šla za učitelem a~koukala skrz okno u~stropu do
	místnosti, kde se akorát jeho děti modlily za Barku. Když to slyšela,
	spadla ze stoličky a~nechtěla se probudit. Učitel ji vzal do pytle
	a~přenesl ji do sklepa chamtivého Kubáta. Kubát nevědomky praštil Barku ve
	sklepě, protože si myslel, že je to zloděj. Ze strachu, že ji zabil, vzal
	ji opět do pytle a~šel na náměstí, kde potkal starostu, taktéž s~pytlem.
	Oba ho pustili a~pak si každý vzal ten druhý pytel. Starosta doma našel
	v~pytli Barku a~polil ji vodou a~ona vstala. Ale ponaučení si z~toho
	nevzala.

\subparagraph{Líná pohádka}
	Líný král měl 3 syny. Ti mu měli říct, kdo se stane králem. Po~3 hodinách
	povídání jim král řekl, že byl líný poslouchat. A~tak zůstal král stejný,
	protože byl líný zemřít.

\subparagraph{Moře, strýčku, proč je slané?}
	Byli dva bratři, chudý Kouba a~bohatý Marek. Kouba měl hodně dětí, ale jednou,
	když neměli co jíst, zašel si k~Markovi pro jídlo. Ten mu neochotně dal půlku
	prasete, ale odhodil ji daleko. Kouba si pro ni došel a~potkal čerta. Ten mu
	řekl, ať jde k~luciferovi a~vymění prase za mlýnek. On to tak udělal a~mlýnek
	si vzal domů. Od té doby si Kouba vedl nejlépe z~celé vesnice, protože mlýnek
	byl kouzelný a~namlel, cokoli si člověk přál. Marek jednou v~noci mlýnek ukradl
	a~odjel s~ním k~moři. Nechal se najmout jako námořník. Jednou při obědě nechal
	mlýnek namlít sůl. Jenže ho neuměl zastavit a~loď se tíhou soli potopila
	i~s~Markem a~mlýnkem. Mlýnek mele dodnes a~proto je moře slané.

\subparagraph{Splněný sen}
	Manželé Loudalovi žili chudě, protože Loudal všechno prosázel, aby vyhrál
	a~koupil si vinohrad. Jednou se mu ve snu zjevila teta Róza a~řekla mu
	o~pokladu a~výherních číslech. Tak jel do města vsadit, ale všechna čísla
	byla o~jedno větší. Tak šel a~jak mu bylo napovězeno, pod pecí našel poklad
	a~koupili s~vinohrad.

\subparagraph{Tři sestry a~jeden prsten}
	Tři sestry, Tylda, Marie a~Bára, dostali každá 8 peněz za kraslice, které ale
	propily. Na cestě domů našli prsten a~určili, že padne té, která svého může
	nejvíce vypeče. Tylda udělala z~manžela opata, Marie jej přiměla si vytrhnout
	zub a~Bára svého přesvědčila, že je mrtvý. Nakonec prsten nepatřil ani jedné.

\subparagraph{Paleček}
	Ševcovi nemohli mít děti. Pak se jim narodil malý chlapeček - Paleček.
	Jednou, když byl donést oběd otci na pole, koupil si ho tam jeden nóbl pán.
	Šel s~ním lesem a~tam ho okradli loupežníci. Využili Palečka ke krádežím
	a~pak ho poslali na sežrání Sněžným žroutům. Ti se ho zprvu báli, ale pak
	ho poslali ke Sněžnému muži. Paleček se stal jeho rádcem, ale řekl mu, že
	potřebuje manželku. Tak si vzal sedmi-mílové boty a~šel zpátky domů. Rodiče
	byli moc rádi. Navštívil toho nóbl pána a~zjistil, že je to král. Ten ho
	požádal, aby mu dělal zpravodaje.

\subparagraph{František Nebojsa}
	Byl to syn muzikanta, který se nikdy nenaučil bát. Až do pěti let nemluvil,
	protože do té doby mu nic nechybělo. Jednou ho jeho otec vzal do hospody,
	aby se tam naučil bát. Počkal si, až bude jedenáct hodin a~to se začala
	objevovat v~hospodě strašidla. Hráli karty a~ten kdo prohrál, toho spálili.
	Když přišla řada na Františka, porval se s~nimi a~tím vysvobodil jednoho
	člověka, který tam byl jen za trest. Ale bát se nenaučil.

\subparagraph{Král měl tři syny}
	Král měl tři syny. Dva poslal pro klobouček s~pérkem sojčím, ale tomu
	třetímu nevěřil. Tak si Honza našetřil, sám se vydal do světa, přivezl otci
	klobouk a~stal se králem. Přes život závodění ale zapomněl na otce, který
	mezitím umřel. Ten mu přenechal klobou. A~tak Honza jel do hospody a~tam si
	našel svoji ženu.

\subparagraph{Tři veteráni}
	Tři vysloužilí vojáci jednou dostali od skřítků tři dary. Klobouk, který
	uměl cokoli vyčarovat, harfu, která uměla vyčarovat a~ovládat třeba i~celé
	vojsko a~bezedný měšec peněz. Tři veteráni si začali žít jako králové.
	Přišli jednou do jednoho království a~dva z~nich se zamilovali do místní
	princezny Bosany, která byla krásná. Řekli jí o~kouzelných darech. Bosana
	spolu se svým otcem jim kouzelné předměty ukradly a~vyhnali je. Jednou
	našli strom frňákovník, na kterém rostla jablka, po kterých člověku pořádně
	narostl nos. Skřítci jim dali tři jablka a~tři hruštičky, po kterých se
	vrátil nos na původní místo. Veteráni pomocí jablek a~hruštiček princeznu
	Bosanu pořádně vytrestali a~vzali si své věci zpět. Skřítkům se ale
	nelíbilo, jak si s~věcmi vedli, nicméně se rozhodli, že jim dají ještě
	jednu šanci.

\subparagraph{Pohádka o~zasloužilém vrabci}
	Byl fousek, který utekl od svého pána. Cestou potkal vrabce, který ho
	dovedl do města, kde se fousek najedl a~napil. Unavený usnul uprostřed
	cesty, kde ho chvíli poté přejel traktorista, i~když ho vrabec přemlouval,
	aby zastavil. Vrabec mu to nedaroval a~podpálil mu traktor. Traktorista se
	začal ohánět kudlou a~v~zápalu vzteku si usekl malíček. Běžel domů
	i~s~vrabcem u~hlavy. Doma se i~se svou \uv{pidlovokou} manželkou začali po
	vrabci ohánět vším, čím se dalo. Traktoristovi se povedlo chytit vrabce do
	ruky. Jeho žena se snažila useknout vrabci hlavu, ale netrefila se a~zabila
	tím manžela. Vrabec odletěl domů. Traktorista měl velký pohřeb.

\subparagraph{O~třech hrbáčích z~Damašku}
	Byli tři bratři. Ti byli ale tak škaredí, že je vyhodili z~Damašku. Šli
	pouští a~rozhodli se jít každý svou cestou. Babekan se oženil a~zbohatnul.
	Když se to dozvěděli jeho bratři, šli za ním, ale ten je vyhodil. Jednou,
	když odešel za přáteli, bratři přišli k~nim do domu, ale on se vrátil dřív
	a~tak je manželka musela schovat do sklepa. Jednou, když zase manžel
	odešel, zavolala hloupého nosiče z~města, aby utopil oba opilé bratry. On
	to tak udělal, ale do vody nevědomky stáhl i~třetího. Jeden bohatý pán,
	který šel okolo, vzal nosiče a~vylovil 3 pytle. Vzal je k~sobě do paláce
	a~ani jeden z~nich se neutopil. Ten bohatý pán dal bratrům hodně peněz, aby
	se z~nich stali úspěšní lidé a~aby se za nimi všichni neotáčeli jen proto,
	že jsou oškliví.

\subparagraph{Rozum \& Štěstí}
	Na lávce se potkali Rozum a~Štěstí. Domluvili se, že Rozum vstoupí do hlavy
	kolemjdoucímu pasáčkovi prasat, aby se mu lépe dařilo. Den na to se Ludvík
	rozhodl odejít z~domova a~stát se královským zahradníkem. Zanedlouho ho
	povýšili na vrchního zahradníka. V~království byla princezna Zasu, která
	nemluvila a~nikdo nevěděl proč. On se rozhodl, že jí to naučí. Princezna mu
	prozradila, že neměla s~kým. Ludvík řekl králi, že ji to naučil a~že by se
	měl stát králem, ale protože neměl důkazy, král ho poslal na popravu. V~tu
	chvíli přišlo na řadu Štěstí. Vyměnilo si místo s~Rozumem a~Ludvíkovi
	zachránilo život. Ludvík a~Zasu se vzali a~on se stal králem.

\subparagraph{Chlap, děd, vnuk, pes a~hrob}
	Příběh sestavený pouze z~jednoslabičných slov aneb chvála češtiny. Byl
	chlap a~ten neustále něco kradl. Jednou, když měl velký hlad, šel krást
	i~ve dne. Přišel k~plotu, přelezl ho a~vešel do zahrady. Tam ho ale
	překvapil pes Rek a~ten štěkal, až vzbudil svého starého pána. Ten přišel
	dolů a~chtěl se prát. Jenže chlap ho zbil a~začal utíkat. Ale vnuk, který
	akorát přiběhl, za ním vystřelil, postřelil ho. Ten se plazil ven
	a~zanedlouho zemřel pod dubem. Rek ho našel a~tím pohádka končí.
}

\parag{Téma a~motiv}{lidské vlastnosti; peníze, lakomost, mazanost, vychytralost, dobrota\dots}
\parag{Časoprostor}{Různé dle pohádky, většinou nějaké království či město}
\parag{Kompoziční výstavba}{19 pohádek; chronologický děj}
\parag{Literární druh a~žánr}{epika, próza; soubor pohádek}

\newpart

\parag{Vypravěč}{er forma, vševědoucí}
\parag{Postavy}{
	\emph{Závislé na příběhu, popsané v~ději}
}
\parag{Vyprávěcí způsob}{nepřímá řeč vypravěče, přímá řeč postav, }
\parag{Typy promluv}{pásma vypravěče, dialogy i~monology}
%\parag{Veršová výstavba}{blankvers}

\newpart

\parag{Jazykové prostředky a~jejich funkce}{spisovná čeština, archaismy,
citoslovce, hovorové výrazy}

\parag{Tropy a~figury a~jejich funkce}{
metafora (\emph{lokaj s~tím letěl}), přirovnání (\emph{mluvil a~mluvil jako
kniha}), personifikace (\emph{les se rozsvítil})
}

\parag{Kontext autorovy tvorby}
\begin{compactitem}
	\item meziválečná tvorba -- 1.~pol 20.~stol
	\item avantgardní divadla (Osvobozené divadlo)
	\item inspirace z~lidové zábavy, kabarety, cirkus\dots
	\item dílo napsáno později v~životě; až po divadelním období
\end{compactitem}

\parag{\getauthor}{
\begin{compactitem}
	\item 1905--1980
	\item hry~-- West pocket revue, Sever proti jihu, Golem
	\item představitel meziválečné divadelní avantgardy
	\item práce s~Voskovcem~-- dva klauni; i~s~Ježkem
	\item představení ve formě revue~-- mluvené slovo, zpěv, balet; forbíny
	\item proti fašismu; 1938 divadlo zavřeno, emigrace do USA
	\item herec a~ředitel divadla ABC, příležitostný filmový herec
	\item tvorba písní~-- David a~Goliáš, Babička Mary, Tmavomodrý svět
\end{compactitem}
}

\parag{Literární/obecně kulturní zasazení}
\begin{description}
\item[Bohumil Hrabal] Ostře sledované vlaky 
\item[Karel Poláček] Bylo nás pět
\item[Josef Čapek] Pejsek a~Kočička
\item[Jaroslav Havlíček] Petrolejové lampy
\end{description}

}
