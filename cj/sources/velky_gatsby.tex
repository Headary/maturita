%kat3
\bookname{Velký Gastby}
\booksubname{}
\bookauthor{Francis Scott Fitzgerald}

\bookcontent{
\newpart

\parag{Děj díla}
Vypravěč příběhu Nick Carraway přijíždí po válce do New Yorku s cílem vydělat
si peníze na burze. Usadí se na Long Islandu v blízkosti mladého zbohatlíka
Jaye Gatsbyho. Brzy se seznámí a Getsby jej začne pravidelně zvát na své
večírky. Na jedné z party se Nick seznámí s Jordan, svou budoucí přítelkyní.
Získává však také možnost lépe poznat Gatsbyho a zjišťuje, že je to velmi
tajemná postava. Kromě tajností kolem obchodování s drogami a alkoholem se Jay
přiznává Nickovi ke své staré lásce bohaté slečně Daisy. Ta se však během války
provdala za jiného muže. Shodou okolností je Daisy Nickova sestřenice. Na
jednom z večírků Daisy přejede nešťastnou náhodou milenku svého manžela –
Myrtle. Myrtlin manžel se chce pomstít a zastřelí Gatsbyho, protože se domnívá,
že to on řídil vůz. Nad smrtí Gatsbyho truchlí jen Nick a Gatsbyho otec, jako
by na něj všichni zapomněli. Daisy s manželem odjíždí a znechucený Nick se
vrací domů.

\parag{Téma a~motiv}
nešťastná láska; láska, štěstí, peníze, podvod, večírky
\parag{Časoprostor}
východní pobřeží USA, New York -- Long Island, léto 1922
\parag{Kompoziční výstavba} % chronologická, retrospektivní...
9 kapitol, chronologický děj, retrospektiva autorova života
\parag{Literární druh a~žánr} % lyrika/epika, poezie/próza...
epika, próza, milostný román
\newpart

\parag{Vypravěč} % er forma, neosobní vypravěč
ich forma, personální vypravěč
\parag{Postavy}
\begin{description}
	\item[Nick Carraway] vypravěč, vzpomíná, prostředník milostného vztahu a
		příběhu, dobrý posluchač, dobrák
	\item[Jay Gatsby] bohatý mladý muž, tajemný, optimistický, lásky k Daisy
	\item[Daisy Buchananová] manželka Toma Buchanana, sestřenice Nicka, milá,
		naivní, přelétavá, povrchní, neví co chce
	\item[Tom Buchanan] manžel Daisy, bohatý, arogantní, nevěrný
	\item[Jordan Bakerová] hráčka golfu, přítelkyně Daisy a Nicka, prostředník
		Daisy a Gatsby
	\item[Myrtle Wilsonová] žena George, milenka Toma, lehká žena
	\item[George Wilson] manžel Myrtle, vlastník autoservisu, chudý, na konci
		zoufalý
\end{description}

\parag{Vyprávěcí způsob} % řeč přímá, polopřímá, nevlastní přímá, nepřímá
přímá, nepřímá řeč

\parag{Typy promluv} % monolog, dialog, pásmo vypravěče
dialogy, pásmo vypravěče
%\parag{Veršová výstavba}

\newpart

\parag{Jazykové prostředky a~jejich funkce} % spisovný/nespisovný jazyk, hovorový
spisovný jazyk, hovorové výrazy
\parag{Tropy a~figury a~jejich funkce} % metafora, metonymie, přirovnání, personifikace
řečnické otázky, elipsy, personifikace, metafora, přirovnání

\parag{Kontext autorovy tvorby}
\begin{compactitem}
	\item meziválečné období
	\item ztracená generace
	\item jazzový věk -- množství novinek a vynálezů
\end{compactitem}

\parag{\getauthor}
\begin{compactitem}
	\item 1896--1940
	\item americký romanopisec, povídkář, scénárista
	\item poválečná ztracená generace
	\item studium v New Jersey a Princetonu
	\item dobrovolník americké armády
	\item mluvčí americké mládeže -- proti pokryteckému životu dospělých
	\item scenárista v Hollywoodu
	\item Něžná je noc, Poslední magnát
\end{compactitem}

\parag{Literární/obecně kulturní zasazení}
\begin{description}
	\item[Romain Rolland] Petr a Lucie
	\item[Ota Pavel] Smrt krásných srnců
	\item[Jaroslav Hašek] Osudy dobrého vojáka Švejka za světové války
\end{description}
}
