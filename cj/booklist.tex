\documentclass[12pt,a4paper]{../books}

\usepackage{tabularx}
\usepackage{dashrule}

\author{Adam Krška}
\title{Seznam knih k maturitě}

\fancyhf{}
\fancyhead[c]{\Large\sffamily\bfseries Gymnázium a střední odborná škola Mikulov,\\příspěvková organizace}

\newcounter{booknum}
\setcounter{booknum}{0}

\newcounter{timenum}
\setcounter{timenum}{0}

\renewcommand\timeperiod[1]{
	\stepcounter{timenum}

	\noindent \alph{timenum})\hspace{2pt} {\bfseries #1}
}

\makeatletter
\renewcommand\printbook[1]{
	\loadbook{#1}\stepcounter{booknum}
	\hspace{0.7cm}\arabic{booknum}.\hspace{2pt}\textbf{\book@name} -- \book@author
}
\makeatother

\begin{document}
\null\vspace{1cm}
\begin{center}
	\sffamily\bfseries
	\huge Seznam literatury k ústní maturitní zkoušce\\
	Školní rok: 2021/2022
\end{center}
\vspace{1cm}

\noindent {\Large Jméno a příjmení, třída: \textbf{Adam Krška, oktáva}}

\vspace{1cm}

\normalsize

\input{bookincludelist.tex}

\newcommand{\dotline}{\hdashrule{6.5cm}{1.5pt}{1.5pt}}

\footnotesize

\vfill

\begin{tabularx}{0.9\textwidth}{>{\centering\arraybackslash}X>{\centering\arraybackslash}X}
V Mikulově dne \hdashrule{4cm}{1.5pt}{1.5pt} & \dotline\\
			   & podpis žáka\\[30pt]
\dotline & \dotline\\
ředitel školy & vyučující\\
\end{tabularx}

\end{document}
